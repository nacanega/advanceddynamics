%% The LaTeX package advanceddynamics - version 1.0 (2023/08/24)
%% advdyndoc-02-frames.tex - Documentation Input File
%%
%% -----------------------------------------------------------------------------
%% Copyright (c) 2023 Nolan Canegallo
%% -----------------------------------------------------------------------------
%%
%% This file may be distributed and/or modified
%%
%%     1.  under the LaTeX Project Public License Version 1.3c and/or
%%     2.  under the GNU Public License Version 3.
%%
%% See the LICENSE.txt for more details.
%%
\section{Frames of Reference}
\label{sec:frames}
\index{reference frames|(}
\index{frames|see {reference frames}}
The commands in this section help the user to define reference frames and their corresponding sets of orthonormal unit vectors.\footnote{Note that my notation differs slightly from the original notation since unit vectors are always marked with a ``hat'' rather than a plain arrow.} Note that all commands in this section should be used inside of a math environment.

\subsection*{\blu{Normal Reference Frames}}
\label{sec:normframes}
\index{reference frames!normal|(}

\entry{% Command Name
    fr
}{% Arguments
    \targsi{point}
}{% Description
    Frame. This command is used to add a bar above a character to allow it to be read as a frame attached to a specified point \meta{point}. Case-insensitive.
}{% Argument Descriptions
    \inputsi{point}{Point; letter (a-z, A-Z)}
}{% Example Description
    Say we want to define an inertial reference frame about the point $O$.
}{% Example Text Arguments
    \comarg{fr}{\targmi{O}}
}{% Example Function Call
    \fr{O}
}

\entry{% Command Name
    frsc
}{% Arguments
    \targsi{frame}
}{% Description
    Small-caps frame. This command is used to define frame \meta{frame} in a subscript or superscript. Case-insensitive.
}{% Argument Descriptions
    \inputsi{frame}{Frame; letter (a-z, A-Z)}
}{% Example Description
    Say we want to define x-component of the vector $\vec{r}$ expressed in the $\fr{O}$ frame.
}{% Example Text Arguments
    \cs{r}\texttt{\string_}\brackets{\texttt{x}\comarg{frsc}{\targmi{O}}}
}{% Example Function Call
    r_{x\frsc{O}}
}

\newpage

\index{unit vectors!normal frame|(}
\index{vectors!unit|see {unit vectors}}
\entry{% Command Name
    ihat 
}{% Arguments
    \targsi{frame}
}{% Description
    I-hat. This command is used to display the $\hat{\imath}$ unit vector for a specified frame \meta{frame}. The bar is automatically added. Case-insensitive.
}{% Argument Descriptions
    \inputsi{frame}{Frame; letter (a-z, A-Z)}
}{% Example Description
    Say we want to express $\hat{\imath}$ for the $\fr{A}$ frame.
}{% Example Text Arguments
    \comarg{ihat}{\targmi{a}}
}{% Example Function Call
    \ihat{a}
}

\entry{% Command Name
    jhat
}{% Arguments
    \targsi{frame}
}{% Description
    J-hat. This command is used to display the $\hat{\jmath}$ unit vector for a specified frame \meta{frame}. The bar is automatically added. Case-insensitive.
}{% Argument Descriptions
    \inputsi{point}{Point; letter (a-z, A-Z)}
}{% Example Description
    Say we want to express $\hat{\jmath}$ for the $\fr{B}$ frame.
}{% Example Text Arguments
    \comarg{jhat}{\targmi{b}}
}{% Example Function Call
    \jhat{b}
}

\newpage

\entry{% Command Name
    khat
}{% Arguments
    \targsi{frame}
}{% Description
    K-hat. This command is used to display the $\hat{k}$ unit vector for a specified frame \meta{frame}. The bar is automatically added. Case-insensitive.
}{% Argument Descriptions
    \inputsi{point}{Point; letter (a-z, A-Z)}
}{% Example Description
    Say we want to express $\hat{k}$ for the $\fr{C}$ frame.
}{% Example Text Arguments
    \comarg{khat}{\targmi{c}}
}{% Example Function Call
    \khat{c}
}
\index{unit vectors!normal frame|)}

\entry{% Command Name
    frDef
}{% Arguments
    \targsi{point}
}{% Description
    Define frame. This command is used define a frame as a point \meta{point} and three orthonormal unit vectors. Case insensitive.
}{% Argument Descriptions
    \inputsi{point}{Point; letter (a-z, A-Z)}
}{% Example Description
    Say we want to define the frame about the point $P$.
}{% Example Text Arguments
    \comarg{frDef}{\targmi{P}}
}{% Example Function Call
    \frDef{P}
}

\newpage

\entry{% Command Name
    frExp
}{% Arguments
    \targsii{frame}{value}
}{% Description
    Expressed in frame. This command is used to show that a matrix or vector quantity \meta{value} is expressed in a given frame \meta{frame}.
}{% Argument Descriptions
    \inputsii{%
        frame}{Frame; letter (a-z, A-Z)}{%
        value}{Vector or matrix quantity}%
}{% Example Description
    Say we want to show that $\vec{v}$ is expressed in the $\fr{B}$ frame.
}{% Example Text Arguments
    \comarg{frExp}{\targmii{B}{\comone{vec}{v}}}
}{% Example Function Call
    \frExp{B}{\vec{v}}
}

\entry{% Command Name
    frVec
}{% Arguments
    \targsiv{frame}{icomp}{jcomp}{kcomp}
}{% Description
    Vector frame components. This command is used to express a vector in terms of its components in each of the frame's unit vectors.
}{% Argument Descriptions
    \inputsiv{%
        frame}{Frame; letter (a-z, A-Z)}{%
        icomp}{i-hat-component expression}{%
        jcomp}{j-hat-component expression}{%
        kcomp}{k-hat-component expression}%
}{% Example Description
    Say we want to express $\frExp{O}{\vec{r}} = \langle x,y,z\rangle$ in its $\fr{O}$ frame components.
}{% Example Text Arguments
    \comarg{frVec}{\targmiv{O}{x}{y}{z}}
}{% Example Function Call
    \frVec{O}{x}{y}{z}
}

\newpage

\entryl{% Command Name
    frTen
}{% Arguments
    \targsx{frame}{ii}{ij}{ik}{ji}{jj}{jk}{ki}{kj}{kk}
}{% Description
    Tensor frame components. This command is used to express a tensor in terms of its components in each of the frame's unit vector combinations. Must be inside a \code{align} or \code{align*} environment.
}{% Argument Descriptions
    \inputsx{%
        frame}{Frame; letter (a-z, A-Z)}{%
        ii}{i-hat-i-hat-component expression}{%
        ij}{i-hat-j-hat-component expression}{%
        ik}{i-hat-k-hat-component expression}{%
        ji}{j-hat-i-hat-component expression}{%
        jj}{j-hat-j-hat-component expression}{%
        jk}{j-hat-k-hat-component expression}{%
        ki}{k-hat-i-hat-component expression}{%
        kj}{k-hat-j-hat-component expression}{%
        kk}{k-hat-k-hat-component expression}%
}{% Example Description
    Say we want to express $\frExp{O}{\tilde{I}}$ in its $\fr{O}$ frame components (say a-i).
}{% Example Text Arguments
    \comarg{frTen}{\targmx{O}{a}{b}{c}{d}{e}{f}{g}{h}{i}}
}{% Example Function Call
    \frTen{O}{a}{b}{c}{d}{e}{f}{g}{h}{i}
}

\newpage

\entry{% Command Name
    frSub
}{% Arguments
    \targsiii{frame}{value}{subsc}
}{% Description
    Subscript and frame. This command is used to add a pre-superscript frame \meta{frame} and specified subscript \meta{subsc} to a value \meta{value}.
}{% Argument Descriptions
    \inputsiii{%
        frame}{Frame; letter (a-z, A-Z)}{%
        value}{Value to apply frame and subscript to}{%
        subsc}{Subscript value}%
}{% Example Description
    Say we want to define the velocity $\vec{v}$ of a satellite expressed in the $\fr{O}$ inertial reference frame.
}{% Example Text Arguments
    \comarg{frSub}{\targmiii{O}{\comone{vec}{v}}{satellite}}
}{% Example Function Call
    \frSub{O}{\vec{v}}{satellite}
}

\entry{% Command Name
    frx
}{% Arguments
    \targsii{frame}{value}
}{% Description
    Frame x-component. This command defines the x-component of value \meta{value} in a specified frame \meta{frame}.
}{% Argument Descriptions
    \inputsii{%
        frame}{Frame; letter (a-z, A-Z)}{%
        value}{The value that we want the x-component of}%
}{% Example Description
    Say we want to define the x-component of the vector $\vec{a}$ expressed in the $\fr{D}$ frame.
}{% Example Text Arguments
    \comarg{frx}{\targmii{D}{\comone{vec}{a}}}
}{% Example Function Call
    \frx{D}{\vec{a}}
}

\newpage

\entry{% Command Name
    fry
}{% Arguments
    \targsii{frame}{value}
}{% Description
    Frame y-component. This command defines the y-component of value \meta{value} in a specified frame \meta{frame}.
}{% Argument Descriptions
    \inputsii{%
        frame}{Frame; letter (a-z, A-Z)}{%
        value}{The value that we want the y-component of}%
}{% Example Description
    Say we want to define the y-component of the vector $\vec{a}$ expressed in the $\fr{D}$ frame.
}{% Example Text Arguments
    \comarg{fry}{\targmii{D}{\comone{vec}{a}}}
}{% Example Function Call
    \fry{D}{\vec{a}}
}

\entry{% Command Name
    frz
}{% Arguments
    \targsii{frame}{value}
}{% Description
    Frame z-component. This command defines the z-component of value \meta{value} in a specified frame \meta{frame}.
}{% Argument Descriptions
    \inputsii{%
        frame}{Frame; letter (a-z, A-Z)}{%
        value}{The value that we want the z-component of}%
}{% Example Description
    Say we want to define the z-component of the vector $\vec{a}$ expressed in the $\fr{D}$ frame.
}{% Example Text Arguments
    \comarg{frz}{\targmii{D}{\comone{vec}{a}}}
}{% Example Function Call
    \frz{D}{\vec{a}}
}
\index{reference frames!normal|)}

\newpage

\subsection*{\blu{Numbered Normal Reference Frames}}
\label{sec:numframes}
\index{reference frames!numbered|(}
This type of frame is similar to a normal frame, except the unit vectors and point also have an associated number.

\entry{% Command Name
    frn
}{% Arguments
    \targsii{point}{num}
}{% Description
    Numbered frame. This command is used to add a bar above a character and a numeric subscript \meta{num} to allow it to be read as a frame attached to a specified point \meta{point}. Case-insensitive.
}{% Argument Descriptions
    \inputsii{%
        point}{Point; letter (a-z, A-Z)}{%
        num}{Number of point}%
}{% Example Description
    Say we want to define an inertial reference frame about the point $A_1$.
}{% Example Text Arguments
    \comarg{fr}{\targmii{A}{1}}
}{% Example Function Call
    \frn{A}{1}
}

\entry{% Command Name
    frnsc
}{% Arguments
    \targsii{point}{num}
}{% Description
    Small-caps numbered frame. This command is used to define a numbered frame in a subscript or superscript. Case-insensitive.
}{% Argument Descriptions
    \inputsi{%
        point}{Point; letter (a-z, A-Z)}{%
        num}{Number of point}%
}{% Example Description
    Say we want to define x-component of the vector $\vec{r}$ expressed in the $\frn{A}{1}$ frame.
}{% Example Text Arguments
    \cs{r}\texttt{\string_}\brackets{\texttt{x}\comarg{frnsc}{\targmii{A}{1}}}
}{% Example Function Call
    r_{x\frnsc{A}{1}}
}

\newpage

\index{unit vectors!numbered frame|(}
\entry{% Command Name
    ihatn
}{% Arguments
    \targsii{frame}{num}
}{% Description
    I-hatn. This command is used to display the $\hat{\imath}$ unit vector for a specified frame \meta{frame} with number \meta{num}. The bar is automatically added. Case-insensitive.
}{% Argument Descriptions
    \inputsii{%
        frame}{Frame; letter (a-z, A-Z)}{%
        num}{Number of point}%
}{% Example Description
    Say we want to express $\hat{\imath}$ for the $\frn{A}{1}$ frame.
}{% Example Text Arguments
    \comarg{ihatn}{\targmii{a}{1}}
}{% Example Function Call
    \ihatn{a}{1}
}

\entry{% Command Name
    jhatn
}{% Arguments
    \targsii{frame}{num}
}{% Description
    J-hatn. This command is used to display the $\hat{\jmath}$ unit vector for a specified frame \meta{frame} with number \meta{num}. The bar is automatically added. Case-insensitive.
}{% Argument Descriptions
    \inputsii{%
        point}{Point; letter (a-z, A-Z)}{%
        num}{Number of point}%
}{% Example Description
    Say we want to express $\hat{\jmath}$ for the $\frn{B}{2}$ frame.
}{% Example Text Arguments
    \comarg{jhatn}{\targmii{b}{2}}
}{% Example Function Call
    \jhatn{b}{2}
}

\newpage

\entry{% Command Name
    khatn
}{% Arguments
    \targsii{frame}{num}
}{% Description
    K-hatn. This command is used to display the $\hat{k}$ unit vector for a specified frame \meta{frame} with number \meta{num}. The bar is automatically added. Case-insensitive.
}{% Argument Descriptions
    \inputsii{%
        point}{Point; letter (a-z, A-Z)}{%
        num}{Number of point}%
}{% Example Description
    Say we want to express $\hat{k}$ for the $\frn{C}{3}$ frame.
}{% Example Text Arguments
    \comarg{khatn}{\targmii{c}{3}}
}{% Example Function Call
    \khatn{c}{3}
}
\index{unit vectors!numbered frame|)}

\entry{% Command Name
    frnDef
}{% Arguments
    \targsii{point}{num}
}{% Description
    Define numbered frame. This command is used define a frame as a numbered point and three orthonormal unit vectors. Case insensitive.
}{% Argument Descriptions
    \inputsii{%
        point}{Point; letter (a-z, A-Z)}{%
        num}{Number of point}%
}{% Example Description
    Say we want to define the frame about the point $P_6$.
}{% Example Text Arguments
    \comarg{frnDef}{\targmii{P}{6}}
}{% Example Function Call
    \frnDef{P}{6}
}

\newpage

\entry{% Command Name
    frnExp
}{% Arguments
    \targsiii{frame}{num}{value}
}{% Description
    Expressed in numbered frame. This command is used to show that a matrix or vector quantity is expressed in a given numbered frame \meta{frame}.
}{% Argument Descriptions
    \inputsiii{%
        frame}{Frame; letter (a-z, A-Z)}{%
        num}{Number of point}{%
        value}{Vector or matrix quantity}%
}{% Example Description
    Say we want to show that $\vec{v}$ is expressed in the $\frn{B}{2}$ frame.
}{% Example Text Arguments
    \comarg{frnExp}{\targmiii{B}{2}{\comone{vec}{v}}}
}{% Example Function Call
    \frnExp{B}{2}{\vec{v}}
}

\entry{% Command Name
    frnVec
}{% Arguments
    \targsv{frame}{num}{icomp}{jcomp}{kcomp}
}{% Description
    Vector numbered frame components. This command is used to express a vector in terms of its components in each of the numbered frame's unit vectors.
}{% Argument Descriptions
    \inputsv{%
        frame}{Frame; letter (a-z, A-Z)}{%
        num}{Number of point}{%
        icomp}{i-hat-component expression}{%
        jcomp}{j-hat-component expression}{%
        kcomp}{k-hat-component expression}%
}{% Example Description
    Say we want to express $\frnExp{A}{1}{\vec{r}} = \langle x,y,z\rangle$ in its $\fr{A}{1}$ frame components.
}{% Example Text Arguments
    \comarg{frnVec}{\targmv{A}{1}{x}{y}{z}}
}{% Example Function Call
    \frnVec{A}{1}{x}{y}{z}
}

\newpage

\entryl{% Command Name
    frnTen
}{% Arguments
    \targsxi{frame}{num}{ii}{ij}{ik}{ji}{jj}{jk}{ki}{kj}{kk}
}{% Description
    Tensor numbered frame components. This command is used to express a tensor in terms of its components in each of the numbered frame's unit vector combinations. Must be inside a \code{align} or \code{align*} environment.
}{% Argument Descriptions
    \inputsxi{%
        frame}{Frame; letter (a-z, A-Z)}{%
        num}{Number of point}{%
        ii}{i-hat-i-hat-component expression}{%
        ij}{i-hat-j-hat-component expression}{%
        ik}{i-hat-k-hat-component expression}{%
        ji}{j-hat-i-hat-component expression}{%
        jj}{j-hat-j-hat-component expression}{%
        jk}{j-hat-k-hat-component expression}{%
        ki}{k-hat-i-hat-component expression}{%
        kj}{k-hat-j-hat-component expression}{%
        kk}{k-hat-k-hat-component expression}%
}{% Example Description
    Say we want to express $\frnExp{A}{1}{\tilde{I}}$ in its $\frn{A}{1}$ frame components (say a-i).
}{% Example Text Arguments
    \comarg{frnTen}{\targmxi{A}{1}{a}{b}{c}{d}{e}{f}{g}{h}{i}}
}{% Example Function Call
    \frnTen{A}{1}{a}{b}{c}{d}{e}{f}{g}{h}{i}
}

\newpage

\entry{% Command Name
    frnSub
}{% Arguments
    \targsiv{frame}{num}{value}{subsc}
}{% Description
    Subscript and numbered frame. This command is used to add a pre-superscript frame and specified subscript to a value.
}{% Argument Descriptions
    \inputsiv{%
        frame}{Frame; letter (a-z, A-Z)}{%
        num}{Number of point}{%
        value}{Value to apply frame and subscript to}{%
        subsc}{Subscript value}%
}{% Example Description
    Say we want to define the velocity of a satellite expressed in the $\frn{A}{1}$ inertial reference frame.
}{% Example Text Arguments
    \comarg{frnSub}{\targmiv{A}{1}{\comone{vec}{v}}{satellite}}
}{% Example Function Call
    \frnSub{A}{1}{\vec{v}}{satellite}
}

\entry{% Command Name
    frnx
}{% Arguments
    \targsiii{frame}{num}{value}
}{% Description
    Numbered frame x-component. This command defines the x-component of value \meta{value} in a specified frame \meta{frame}.
}{% Argument Descriptions
    \inputsiii{%
        frame}{Frame; letter (a-z, A-Z)}{%
        num}{Number of point}{%
        value}{The value that we want the x-component of}%
}{% Example Description
    Say we want to define the x-component of the vector $\vec{a}$ expressed in the $\frn{D}{4}$ frame.
}{% Example Text Arguments
    \comarg{frnx}{\targmiii{D}{4}{\comone{vec}{a}}}
}{% Example Function Call
    \frnx{D}{4}{\vec{a}}
}

\newpage

\entry{% Command Name
    frny
}{% Arguments
    \targsiii{frame}{num}{value}
}{% Description
    Numbered frame y-component. This command defines the y-component of value \meta{value} in a specified frame \meta{frame}.
}{% Argument Descriptions
    \inputsiii{%
        frame}{Frame; letter (a-z, A-Z)}{%
        num}{Number of point}{%
        value}{The value that we want the y-component of}%
}{% Example Description
    Say we want to define the y-component of the vector $\vec{a}$ expressed in the $\frn{D}{4}$ frame.
}{% Example Text Arguments
    \comarg{frny}{\targmiii{D}{4}{\comone{vec}{a}}}
}{% Example Function Call
    \frny{D}{4}{\vec{a}}
}

\entry{% Command Name
    frnz
}{% Arguments
    \targsiii{frame}{num}{value}
}{% Description
    Numbered frame z-component. This command defines the z-component of value \meta{value} in a specified frame \meta{frame}.
}{% Argument Descriptions
    \inputsiii{%
        frame}{Frame; letter (a-z, A-Z)}{%
        num}{Number of point}{%
        value}{The value that we want the z-component of}%
}{% Example Description
    Say we want to define the z-component of the vector $\vec{a}$ expressed in the $\frn{D}{4}$ frame.
}{% Example Text Arguments
    \comarg{frnz}{\targmiii{D}{4}{\comone{vec}{a}}}
}{% Example Function Call
    \frnz{D}{4}{\vec{a}}
}
\index{reference frames!numbered|)}

\newpage

\subsection*{\blu{Special Reference Frames}}
\label{sec:specframes}
\index{reference frames!special|(}
This type of frame is similar to a normal frame, except the unit vectors are numbered lowercase versions of the specified point.

\index{unit vectors!special frame|(}
\entry{% Command Name
    uveca
}{% Arguments
    \targsi{frame}
}{% Description
    Unit vector 1. This command is used to display the first unit vector for a specified frame \meta{frame}. The bar is automatically added. Case-insensitive.
}{% Argument Descriptions
    \inputsi{frame}{Frame; letter (a-z, A-Z)}
}{% Example Description
    Say we want to express the first unit vector for the $\fr{A}$ frame.
}{% Example Text Arguments
    \comarg{uveca}{\targmi{a}}
}{% Example Function Call
    \uveca{a}
}

\entry{% Command Name
    uvecb
}{% Arguments
    \targsi{frame}
}{% Description
    Unit vector 2. This command is used to display the second unit vector for a specified frame \meta{frame}. The bar is automatically added. Case-insensitive.
}{% Argument Descriptions
    \inputsi{point}{Point; letter (a-z, A-Z)}
}{% Example Description
    Say we want to express the second unit vector for the $\fr{B}$ frame.
}{% Example Text Arguments
    \comarg{uvecb}{\targmi{b}}
}{% Example Function Call
    \uvecb{b}
}

\newpage

\entry{% Command Name
    uvecc
}{% Arguments
    \targsi{frame}
}{% Description
    Unit vector 3. This command is used to display the third unit vector for a specified frame \meta{frame}. The bar is automatically added. Case-insensitive.
}{% Argument Descriptions
    \inputsi{point}{Point; letter (a-z, A-Z)}
}{% Example Description
    Say we want to express the third unit vector for the $\fr{C}$ frame.
}{% Example Text Arguments
    \comarg{uvecc}{\targmi{c}}
}{% Example Function Call
    \uvecc{c}
}
\index{unit vectors!special frame|)}

\entry{% Command Name
    fruDef
}{% Arguments
    \targsi{point}
}{% Description
    Define special frame. This command is used define a special frame as a point and three orthonormal unit vectors. Case insensitive.
}{% Argument Descriptions
    \inputsi{point}{Point; letter (a-z, A-Z)}
}{% Example Description
    Say we want to define the special frame about the point $P$.
}{% Example Text Arguments
    \comarg{fruDef}{\targmi{P}}
}{% Example Function Call
    \fruDef{P}
}

\newpage

\entry{% Command Name
    fruVec
}{% Arguments
    \targsiv{frame}{comp1}{comp2}{comp3}
}{% Description
    Vector special frame components. This command is used to express a vector in terms of its components in each of the special frame's unit vectors.
}{% Argument Descriptions
    \inputsiv{%
        frame}{Frame; letter (a-z, A-Z)}{%
        comp1}{1-component expression}{%
        comp2}{2-component expression}{%
        comp3}{3-component expression}%
}{% Example Description
    Say we want to express $\frExp{F}{\vec{r}} = \langle x,y,z\rangle$ in its $\fr{F}$ frame components.
}{% Example Text Arguments
    \comarg{fruVec}{\targmiv{F}{x}{y}{z}}
}{% Example Function Call
    \fruVec{F}{x}{y}{z}
}

\newpage

\entryl{% Command Name
    fruTen
}{% Arguments
    \targsx{frame}{c11}{c12}{c13}{c21}{c22}{c23}{c31}{c32}{c33}
}{% Description
    Tensor special frame components. This command is used to express a tensor in terms of its components in each of the special frame's unit vector combinations. Must be inside a \code{align} or \code{align*} environment.
}{% Argument Descriptions
    \inputsx{%
        frame}{Frame; letter (a-z, A-Z)}{%
        c11}{1-1-component expression}{%
        c12}{1-2-component expression}{%
        c13}{1-3-component expression}{%
        c21}{2-1-component expression}{%
        c22}{2-2-component expression}{%
        c23}{2-3-component expression}{%
        c31}{3-1-component expression}{%
        c32}{3-2-component expression}{%
        c33}{3-3-component expression}%
}{% Example Description
    Say we want to express $\frExp{F}{\tilde{I}}$ in its $\fr{F}$ frame components (say a-i).
}{% Example Text Arguments
    \comarg{fruTen}{\targmx{F}{a}{b}{c}{d}{e}{f}{g}{h}{i}}
}{% Example Function Call
    \fruTen{F}{a}{b}{c}{d}{e}{f}{g}{h}{i}
}
\index{reference frames!special|)}
\index{reference frames|)}