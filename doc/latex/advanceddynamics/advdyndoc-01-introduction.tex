%% The LaTeX package advanceddynamics - version 1.0 (2023/08/24)
%% advdyndoc-01-introduction.tex - Documentation Input File
%%
%% -----------------------------------------------------------------------------
%% Copyright (c) 2023 Nolan Canegallo
%% -----------------------------------------------------------------------------
%%
%% This file may be distributed and/or modified
%%
%%     1.  under the LaTeX Project Public License Version 1.3c and/or
%%     2.  under the GNU Public License Version 3.
%%
%% See the LICENSE.txt for more details.
%%
\section{Introduction}
\label{sec:intro}
This document serves as the documentation for the advanced dynamics notation \LaTeX\ package \code{advanceddynamics} which provides a set of macros and commands to easily create text in the notation of the NCSU graduate courses \textit{MAE 511 - Advanced Dynamics I} and \textit{MAE 789 - Advanced Dynamics II} taught by Dr. Mazzoleni.

This document will outline all of the available commands and provide some examples of the typesetting.

\subsection{Required Packages}
\label{sec:requirements}
\index{package!required}
There are 6 required packages to use \code{advanceddynamics}. They are:
\begin{enumerate}[noitemsep]
	\item \code{accents} for custom bar \cite{pack:accents}
	\item \code{amsmath} for math notation \cite{pack:amsmath}
	\item \code{amssymb} for math symbols \cite{pack:amssymb}
	\item \code{graphicx} for scaling subscripts and superscripts \cite{pack:graphicx}
	\item \code{mathtools} for additional math functionality \cite{pack:mathtools}
	\item \code{tensor} for prescripts \cite{pack:tensor}
\end{enumerate}

These packages will be automatically imported when using \code{advanceddynamics}. However, importing these packages first is a good way to ensure that they load correctly, especially since this is my first \LaTeX\ package. Note that \code{mathtools} already includes the \code{amsmath} package, so you can omit its import if desired.

This can be done with the following lines before your document's \comone{begin}{document}:
\begin{indentList}
	\item \comone{usepackage}{accents}
	\item \comone{usepackage}{amsmath}
    \item \comone{usepackage}{amsym}
	\item \comone{usepackage}{graphicx}
	\item \comone{usepackage}{mathtools}
	\item \comone{usepackage}{tensor}
\end{indentList}

\subsection{Obtaining the Package}
\label{sec:download}
\index{download}
The \code{advanceddynamics} package and documentation are available for download at:
\begin{indentList}
	\item \url{https://github.com/nacanega/advanceddynamics}
\end{indentList}

\subsection{Using the Package}
\index{package!usage}
\label{sec:usage}
You can use \code{advanceddynamics} by including the following line before your document's \comone{begin}{document}:
\begin{indentList}
	\item \comone{usepackage}{advanceddynamics}
\end{indentList}

Note that you will need to ensure that the file \code{advanceddynamics.sty} is either located in the same location as your ``.tex'' files or placed with your other \LaTeX\ distribution's local packages. More detailed information can be found in Section \ref{sec:install}.