%% The LaTeX package advanceddynamics - version 1.0 (2023/08/24)
%% advdyndoc-05-differentiation.tex - Documentation Input File
%%
%% -----------------------------------------------------------------------------
%% Copyright (c) 2023 Nolan Canegallo
%% -----------------------------------------------------------------------------
%%
%% This file may be distributed and/or modified
%%
%%     1.  under the LaTeX Project Public License Version 1.3c and/or
%%     2.  under the GNU Public License Version 3.
%%
%% See the LICENSE.txt for more details.
%%
\section{Differentiation}
\label{sec:diff}
\index{derivative|(}
This section includes the commands needed to typeset most derivatives and their calculation. Note that all commands in this section should be used inside of a math environment.

\subsection*{\blu{Time Derivatives}}
\label{sec:timediff}
\index{derivative!time|(}

\entry{% Command Name
    derI
}{% Arguments
    \targsii{frame}{expr}
}{% Description
    1st time derivative. This command is used to take the first time derivative of \meta{expr} in the \meta{frame} frame.
}{% Argument Descriptions
    \inputsii{%
        frame}{Frame; letter (a-z, A-Z)}{%
        expr}{Expression; expression}%
}{% Example Description
    Say we want to define the first time derivative of position, $\vec{r}$, in the $\fr{A}$ frame.
}{% Example Text Arguments
    \comarg{derI}{\targmii{a}{\comone{vec}{r}\cs{,}}}
}{% Example Function Call
    \derI{a}{\vec{r}\,}
}

\entry{% Command Name
    derII
}{% Arguments
    \targsii{frame}{expr}
}{% Description
    2nd time derivative. This command is used to take the second time derivative of \meta{expr} in the \meta{frame} frame.
}{% Argument Descriptions
    \inputsii{%
        frame}{Frame; letter (a-z, A-Z)}{%
        expr}{Expression; expression}%
}{% Example Description
    Say we want to define the second time derivative of position, $\vec{r}$, in the $\fr{A}$ frame.
}{% Example Text Arguments
    \comarg{derII}{\targmii{a}{\comone{vec}{r}\cs{,}}}
}{% Example Function Call
    \derII{a}{\vec{r}\,}
}

\newpage

\entry{% Command Name
    derN
}{% Arguments
    \targsiii{frame}{expr}{n}
}{% Description
    nth time derivative. This command is used to take the nth time derivative of \meta{expr} in the \meta{frame} frame.
}{% Argument Descriptions
    \inputsiii{%
        frame}{Frame; letter (a-z, A-Z)}{%
        expr}{Expression; expression}{%
        n}{Number, positive integer}%
}{% Example Description
    Say we want to define the 100th time derivative of position, $\vec{r}$, in the $\fr{A}$ frame.
}{% Example Text Arguments
    \comarg{derN}{\targmiii{a}{\comone{vec}{r}\cs{,}}{100}}
}{% Example Function Call
    \derN{a}{\vec{r}\,}{100}
}

\index{theorem|see {transport theorem}}
\index{transport theorem!first-order!time}
\entry{% Command Name
    tranI
}{% Arguments
    \targsiii{dframe}{eframe}{expr}
}{% Description
    First-order transport theorem. This command is used to show how to take the first \meta{dframe}-frame time derivative of \meta{expr} which is expressed in the \meta{eframe}-frame.
}{% Argument Descriptions
    \inputsiii{%
        dframe}{Derivative frame; letter (a-z, A-Z)}{%
        eframe}{Expressed frame; letter (a-z, A-Z)}{%
        expr}{Expression; expression}%
}{% Example Description
    Say we want to find the first $\fr{b}$ frame time derivative of the vector $\vec{q}$, which is expressed in the $\fr{a}$ frame.
}{% Example Text Arguments
    \comarg{tranI}{\targmiii{b}{a}{\comone{vec}{q}\cs{,}}}
}{% Example Function Call
    \tranI{b}{a}{\vec{q}\,}
}

\newpage

\index{transport theorem!second-order!time}
\entry{% Command Name
    tranII
}{% Arguments
    \targsiii{dframe}{eframe}{expr}
}{% Description
    Second-order transport theorem. This command is used to show how to take the second \meta{dframe}-frame time derivative of \meta{expr} which is expressed in the \meta{eframe}-frame.
}{% Argument Descriptions
    \inputsiii{%
        dframe}{Derivative frame; letter (a-z, A-Z)}{%
        eframe}{Expressed frame; letter (a-z, A-Z)}{%
        expr}{Expression; expression}%
}{% Example Description
    Say we want to find the second $\fr{b}$ frame time derivative of the vector $\vec{q}$, which is expressed in the $\fr{a}$ frame.
}{% Example Text Arguments
    \comarg{tranII}{\targmiii{b}{a}{\comone{vec}{q}\cs{,}}}
}{% Example Function Call
    \tranII{b}{a}{\vec{q}\,}
}
\index{derivative!time|)}

\newpage

\subsection*{\blu{Partial Derivatives}}
\label{sec:partdiff}
\index{derivative!partial|(}

\entry{% Command Name
    pderI
}{% Arguments
    \targsiii{frame}{var}{expr}
}{% Description
    1st partial derivative. This command is used to take the first partial derivative with respect to \meta{var} of \meta{expr} in the \meta{frame} frame.
}{% Argument Descriptions
    \inputsiii{%
        frame}{Frame; letter (a-z, A-Z)}{%
        var}{Variable; symbol}{%
        expr}{Expression; expression}%
}{% Example Description
    Say we want to define the first partial derivative of velocity, $\vec{v}$, with respect to $\theta$ in the $\fr{B}$ frame.
}{% Example Text Arguments
    \comarg{pderI}{\targmiii{b}{\cs{theta}}{\comone{vec}{v}\cs{,}}}
}{% Example Function Call
    \pderI{b}{\theta}{\vec{v}\,}
}

\entry{% Command Name
    pderII
}{% Arguments
    \targsiii{frame}{var}{expr}
}{% Description
    2nd partial derivative. This command is used to take the second partial derivative with respect to \meta{var} of \meta{expr} in the \meta{frame} frame.
}{% Argument Descriptions
    \inputsiii{%
        frame}{Frame; letter (a-z, A-Z)}{%
        var}{Variable; symbol}{%
        expr}{Expression; expression}%
}{% Example Description
    Say we want to define the second partial derivative of velocity, $\vec{v}$, with respect to $\theta$ in the $\fr{B}$ frame.
}{% Example Text Arguments
    \comarg{pderII}{\targmiii{b}{\cs{theta}}{\comone{vec}{v}\cs{,}}}
}{% Example Function Call
    \pderII{b}{\theta}{\vec{v}\,}
}

\newpage

\entry{% Command Name
    pderN
}{% Arguments
    \targsiv{frame}{var}{expr}{n}
}{% Description
    nth partial derivative. This command is used to take the nth partial derivative with respect to \meta{var} of \meta{expr} in the \meta{frame} frame.
}{% Argument Descriptions
    \inputsiv{%
        frame}{Frame; letter (a-z, A-Z)}{%
        var}{Variable; symbol}{%
        expr}{Expression; expression}{%
        n}{Number, positive integer}%
}{% Example Description
    Say we want to define the 30th partial derivative of velocity, $\vec{v}$, with respect to $\theta$ in the $\fr{B}$ frame.
}{% Example Text Arguments
    \comarg{pderN}{\targmiv{b}{\cs{theta}}{\comone{vec}{v}\cs{,}}{30}}
}{% Example Function Call
    \pderN{b}{\theta}{\vec{v}\,}{30}
}

\index{transport theorem!first-order!partial}
\entry{% Command Name
    ptranI
}{% Arguments
    \targsiv{dframe}{eframe}{var}{expr}
}{% Description
    First-order partial derivative transport theorem. This command is used to show how to take the first partial \meta{dframe}-frame derivative with respect to \meta{var} of \meta{expr} which is expressed in the \meta{eframe}-frame.
}{% Argument Descriptions
    \inputsiv{%
        dframe}{Derivative frame; letter (a-z, A-Z)}{%
        eframe}{Expressed frame; letter (a-z, A-Z)}{%
        var}{Variable; symbol}{%
        expr}{Expression; expression}%
}{% Example Description
    Say we want to find the first partial $\fr{b}$ frame derivative with respect to $\phi$ of the vector $\vec{q}$, which is expressed in the $\fr{a}$ frame.
}{% Example Text Arguments
    \comarg{ptranI}{\targmiv{b}{a}{\cs{phi}}{\comone{vec}{q}\cs{,}}}
}{% Example Function Call
    \ptranI{b}{a}{\phi}{\vec{q}\,}
}
\index{derivative!partial|)}

\newpage

\subsection*{\blu{Total Derivatives}}
\label{sec:totdiff}
\index{derivative!total|(}

\entry{% Command Name
    tderI
}{% Arguments
    \targsiii{frame}{var}{expr}
}{% Description
    1st total derivative. This command is used to take the first total derivative with respect to \meta{var} of \meta{expr} in the \meta{frame} frame.
}{% Argument Descriptions
    \inputsiii{%
        frame}{Frame; letter (a-z, A-Z)}{%
        var}{Variable; symbol}{%
        expr}{Expression; expression}%
}{% Example Description
    Say we want to define the first total derivative of velocity, $\vec{v}$, with respect to $\psi$ in the $\fr{O}$ frame.
}{% Example Text Arguments
    \comarg{tderI}{\targmiii{o}{\cs{psi}}{\comone{vec}{v}\cs{,}}}
}{% Example Function Call
    \tderI{o}{\psi}{\vec{v}\,}
}

\entry{% Command Name
    tderII
}{% Arguments
    \targsiii{frame}{var}{expr}
}{% Description
    2nd total derivative. This command is used to take the second total derivative with respect to \meta{var} of \meta{expr} in the \meta{frame} frame.
}{% Argument Descriptions
    \inputsiii{%
        frame}{Frame; letter (a-z, A-Z)}{%
        var}{Variable; symbol}{%
        expr}{Expression; expression}%
}{% Example Description
    Say we want to define the second total derivative of velocity, $\vec{v}$, with respect to $\psi$ in the $\fr{O}$ frame.
}{% Example Text Arguments
    \comarg{tderII}{\targmiii{o}{\cs{psi}}{\comone{vec}{v}\cs{,}}}
}{% Example Function Call
    \tderII{o}{\psi}{\vec{v}\,}
}

\newpage

\entry{% Command Name
    tderN
}{% Arguments
    \targsiv{frame}{var}{expr}{n}
}{% Description
    nth total derivative. This command is used to take the nth total derivative with respect to \meta{var} of \meta{expr} in the \meta{frame} frame.
}{% Argument Descriptions
    \inputsiv{%
        frame}{Frame; letter (a-z, A-Z)}{%
        var}{Variable; symbol}{%
        expr}{Expression; expression}{%
        n}{Number, positive integer}%
}{% Example Description
    Say we want to define the 30th total derivative of velocity, $\vec{v}$, with respect to $\psi$ in the $\fr{O}$ frame.
}{% Example Text Arguments
    \comarg{tderN}{\targmiv{o}{\cs{psi}}{\comone{vec}{v}\cs{,}}{30}}
}{% Example Function Call
    \tderN{o}{\psi}{\vec{v}\,}{30}
}

\index{transport theorem!first-order!total}
\entry{% Command Name
    ttranI
}{% Arguments
    \targsiv{dframe}{eframe}{var}{expr}
}{% Description
    First-order total derivative transport theorem. This command is used to show how to take the first total \meta{dframe}-frame derivative with respect to \meta{var} of \meta{expr} which is expressed in the \meta{eframe}-frame.
}{% Argument Descriptions
    \inputsiv{%
        dframe}{Derivative frame; letter (a-z, A-Z)}{%
        eframe}{Expressed frame; letter (a-z, A-Z)}{%
        var}{Variable; symbol}{%
        expr}{Expression; expression}%
}{% Example Description
    Say we want to find the first total $\fr{o}$ frame derivative with respect to $\phi$ of the vector $\vec{q}$, which is expressed in the $\fr{a}$ frame.
}{% Example Text Arguments
    \comarg{ttranI}{\targmiv{o}{a}{\cs{phi}}{\comone{vec}{q}\cs{,}}}
}{% Example Function Call
    \ttranI{o}{a}{\phi}{\vec{q}\,}
}
\index{derivative!total|)}
\index{derivative|)}
\index{differentiation|see {derivative}}