%% The LaTeX package advanceddynamics - version 1.0 (2023/08/24)
%% advdyndoc-08-equations.tex - Documentation Input File
%%
%% -----------------------------------------------------------------------------
%% Copyright (c) 2023 Nolan Canegallo
%% -----------------------------------------------------------------------------
%%
%% This file may be distributed and/or modified
%%
%%     1.  under the LaTeX Project Public License Version 1.3c and/or
%%     2.  under the GNU Public License Version 3.
%%
%% See the LICENSE.txt for more details.
%%
\section{Equations}
\label{sec:eqs}
\index{equations|(}
This section includes the commands needed to typeset most of the fundamental equations from advanced dynamics. Note that all commands in this section should be used inside of a math environment and a display math environment should be used for the best appearance.

Additionally, it should be mentioned that the descriptions of the functions may not be enough to understand all of these definitions. This is partly intentional since it is based on a notation from a course and fully comprehending the equations would require one taking the course themself. However, the descriptions may be updated to be more detailed in the future.

\subsection*{\blu{Newton's Second Law}}
\label{sec:newtonII}

\index{equations!Newton's second law|(}
\entryl{% Command Name
    NewtonII
}{% Arguments
    { }
}{% Description
    Newton II. Shows Newton's Second Law for a system of $k$ rigid bodies and $n$ particles. Assumes Newton III.
}{% Argument Descriptions
    \textit{No input arguments.}
}{% Example Description
    Show the generic form of Newton's Second Law when Newton III applies.
}{% Example Text Arguments
    \cs{NewtonII}
}{% Example Function Call
    \NewtonII
}

\newpage

\entryl{% Command Name
    NewtonIIpart
}{% Arguments
    { }
}{% Description
    Newton II for particles. Shows Newton's Second Law for a system of $n$ particles. Assumes Newton III.
}{% Argument Descriptions
    \textit{No input arguments.}
}{% Example Description
    Show the particle form of Newton's Second Law.
}{% Example Text Arguments
    \cs{NewtonIIpart}
}{% Example Function Call
    \NewtonIIpart
}

\entryl{% Command Name
    NewtonIIrigp
}{% Arguments
    { }
}{% Description
    Newton II for rigid body. Shows Newton's Second Law for a system of particles as a rigid body. Assumes Newton III.
}{% Argument Descriptions
    \textit{No input arguments.}
}{% Example Description
    Show the rigid body form of Newton's Second Law.
}{% Example Text Arguments
    \cs{NewtonIIrigp}
}{% Example Function Call
    \NewtonIIrigp
}

\newpage

\entryl{% Command Name
    NewtonIIgen
}{% Arguments
    { }
}{% Description
    Generalized Newton II. Shows Newton's Second Law for a generic system of $n$ particles.
}{% Argument Descriptions
    \textit{No input arguments.}
}{% Example Description
    Show the generic form of Newton's Second Law.
}{% Example Text Arguments
    \cs{NewtonIIgen}
}{% Example Function Call
    \NewtonIIgen
}
\index{equations!Newton's second law|)}

\subsection*{\blu{Full Inertia Equations}}
\label{sec:inereqs}
\index{equations!inertia|(}

\entryl{% Command Name
    inerTenDef
}{% Arguments
    \targsi{point}
}{% Description
    Inertia tensor definition. Defines the inertia tensor about a specified point, \meta{point}.
}{% Argument Descriptions
    \inputsi{point}{Point; expression}
}{% Example Description
    Say we want to define the inertia tensor about point $A$.
}{% Example Text Arguments
    \comarg{inerTenDef}{\targmi{\comone{smca}{A}}}
}{% Example Function Call
    \inerTenDef{\smca{A}}
}

\newpage

\index{theorem!parallel axis!tensor}
\entryd{% Command Name
    parAxisTen
}{% Arguments
    \targsi{point}
}{% Description
    Parallel axis theorem tensor definition. Expression of the parallel axis theorem in tensor form about a specified point $A$.
}{% Argument Descriptions
    \inputsi{point}{Point; expression}
}{% Example Description
    Say we want to define the inertia tensor about point $A$ using the parallel axis theorem.
}{% Example Text Arguments
    \comarg{parAxisTen}{\targmi{\comone{smca}{A}}}
}{% Example Function Call
    \parAxisTen{\smca{A}}
}

\entryd{% Command Name
    rRel
}{% Arguments
    \targsiii{frame}{sys}{point}
}{% Description
    Relative displacement. This is the relative displacement vector for the inertia tensor of the system \meta{sys} relative to the point \meta{point} with respect to the frame \meta{frame}. 
}{% Argument Descriptions
    \inputsiii{%
    	frame}{Frame; letter (a-z, A-Z)}{%
    	sys}{System name; expression}{%
    	point}{Reference point; expression}%
}{% Example Description
    Say we want to define the relative displacement of a mass element $m_i$ relative to the point $B$ in the body frame $\fr{B}$
}{% Example Text Arguments
    \comarg{rRel}{\targmiii{b}{m\string_i}{\comone{smca}{B}}}
}{% Example Function Call
    \rRel{b}{m_i}{\smca{B}}
}

\newpage

\entryl{% Command Name
    IxxSum
}{% Arguments
    \targsii{frame}{point}
}{% Description
    $I_{xx}$ term sum definition. Defines the $I_{xx}$ term of the inertia tensor about a point \meta{point} in frame \meta{frame}.
}{% Argument Descriptions
    \inputsii{%
    	frame}{Frame; letter (a-z, A-Z)}{%
    	point}{Point; expression}%
}{% Example Description
    Say we want to define the $I_{xx}$ term of the inertia tensor about point $A$ in frame $\fr{B}$.
}{% Example Text Arguments
    \comarg{IxxSum}{\targmii{b}{\comone{smca}{a}}}
}{% Example Function Call
    \IxxSum{b}{\smca{a}}
}

\entryl{% Command Name
    IxySum
}{% Arguments
    \targsii{frame}{point}
}{% Description
    $I_{xy}$ term sum definition. Defines the $I_{xy}$ term of the inertia tensor about a point \meta{point} in frame \meta{frame}.
}{% Argument Descriptions
    \inputsii{%
    	frame}{Frame; letter (a-z, A-Z)}{%
    	point}{Point; expression}%
}{% Example Description
    Say we want to define the $I_{xy}$ term of the inertia tensor about point $A$ in frame $\fr{B}$.
}{% Example Text Arguments
    \comarg{IxySum}{\targmii{b}{\comone{smca}{a}}}
}{% Example Function Call
    \IxySum{b}{\smca{a}}
}

\newpage

\entryl{% Command Name
    IxzSum
}{% Arguments
    \targsii{frame}{point}
}{% Description
    $I_{xz}$ term sum definition. Defines the $I_{xz}$ term of the inertia tensor about a point \meta{point} in frame \meta{frame}.
}{% Argument Descriptions
    \inputsii{%
    	frame}{Frame; letter (a-z, A-Z)}{%
    	point}{Point; expression}%
}{% Example Description
    Say we want to define the $I_{xz}$ term of the inertia tensor about point $A$ in frame $\fr{B}$.
}{% Example Text Arguments
    \comarg{IxzSum}{\targmii{b}{\comone{smca}{a}}}
}{% Example Function Call
    \IxzSum{b}{\smca{a}}
}

\entryl{% Command Name
    IyxSum
}{% Arguments
    \targsii{frame}{point}
}{% Description
    $I_{yx}$ term sum definition. Defines the $I_{yx}$ term of the inertia tensor about a point \meta{point} in frame \meta{frame}.
}{% Argument Descriptions
    \inputsii{%
    	frame}{Frame; letter (a-z, A-Z)}{%
    	point}{Point; expression}%
}{% Example Description
    Say we want to define the $I_{yx}$ term of the inertia tensor about point $A$ in frame $\fr{B}$.
}{% Example Text Arguments
    \comarg{IyxSum}{\targmii{b}{\comone{smca}{a}}}
}{% Example Function Call
    \IyxSum{b}{\smca{a}}
}

\newpage

\entryl{% Command Name
    IyySum
}{% Arguments
    \targsii{frame}{point}
}{% Description
    $I_{yy}$ term sum definition. Defines the $I_{yy}$ term of the inertia tensor about a point \meta{point} in frame \meta{frame}.
}{% Argument Descriptions
    \inputsii{%
    	frame}{Frame; letter (a-z, A-Z)}{%
    	point}{Point; expression}%
}{% Example Description
    Say we want to define the $I_{yy}$ term of the inertia tensor about point $A$ in frame $\fr{B}$.
}{% Example Text Arguments
    \comarg{IyySum}{\targmii{b}{\comone{smca}{a}}}
}{% Example Function Call
    \IyySum{b}{\smca{a}}
}

\entryl{% Command Name
    IyzSum
}{% Arguments
    \targsii{frame}{point}
}{% Description
    $I_{yz}$ term sum definition. Defines the $I_{yz}$ term of the inertia tensor about a point \meta{point} in frame \meta{frame}.
}{% Argument Descriptions
    \inputsii{%
    	frame}{Frame; letter (a-z, A-Z)}{%
    	point}{Point; expression}%
}{% Example Description
    Say we want to define the $I_{yz}$ term of the inertia tensor about point $A$ in frame $\fr{B}$.
}{% Example Text Arguments
    \comarg{IyzSum}{\targmii{b}{\comone{smca}{a}}}
}{% Example Function Call
    \IyzSum{b}{\smca{a}}
}

\newpage

\entryl{% Command Name
    IzxSum
}{% Arguments
    \targsii{frame}{point}
}{% Description
    $I_{zx}$ term sum definition. Defines the $I_{zx}$ term of the inertia tensor about a point \meta{point} in frame \meta{frame}.
}{% Argument Descriptions
    \inputsii{%
    	frame}{Frame; letter (a-z, A-Z)}{%
    	point}{Point; expression}%
}{% Example Description
    Say we want to define the $I_{zx}$ term of the inertia tensor about point $A$ in frame $\fr{B}$.
}{% Example Text Arguments
    \comarg{IzxSum}{\targmii{b}{\comone{smca}{a}}}
}{% Example Function Call
    \IzxSum{b}{\smca{a}}
}

\entryl{% Command Name
    IzySum
}{% Arguments
    \targsii{frame}{point}
}{% Description
    $I_{zy}$ term sum definition. Defines the $I_{zy}$ term of the inertia tensor about a point \meta{point} in frame \meta{frame}.
}{% Argument Descriptions
    \inputsii{%
    	frame}{Frame; letter (a-z, A-Z)}{%
    	point}{Point; expression}%
}{% Example Description
    Say we want to define the $I_{zy}$ term of the inertia tensor about point $A$ in frame $\fr{B}$.
}{% Example Text Arguments
    \comarg{IzySum}{\targmii{b}{\comone{smca}{a}}}
}{% Example Function Call
    \IzySum{b}{\smca{a}}
}

\newpage

\entryl{% Command Name
    IzzSum
}{% Arguments
    \targsii{frame}{point}
}{% Description
    $I_{zz}$ term sum definition. Defines the $I_{zz}$ term of the inertia tensor about a point \meta{point} in frame \meta{frame}.
}{% Argument Descriptions
    \inputsii{%
    	frame}{Frame; letter (a-z, A-Z)}{%
    	point}{Point; expression}%
}{% Example Description
    Say we want to define the $I_{zz}$ term of the inertia tensor about point $A$ in frame $\fr{B}$.
}{% Example Text Arguments
    \comarg{IzzSum}{\targmii{b}{\comone{smca}{a}}}
}{% Example Function Call
    \IzzSum{b}{\smca{a}}
}

\index{theorem!parallel axis!matrix}
\entryl{% Command Name
    parAxis
}{% Arguments
    \targsii{frame}{point}
}{% Description
    Parallel axis theorem in matrix form. Defines the terms of the inertia tensor about a point \meta{point} in frame \meta{frame} using the parallel axis theorem.
}{% Argument Descriptions
    \inputsii{%
    	frame}{Frame; letter (a-z, A-Z)}{%
    	point}{Point; expression}%
}{% Example Description
    Say we want to define the inertia tensor in matrix form about point $A$ in frame $\fr{B}$ using the parallel axis theorem.
}{% Example Text Arguments
    \comarg{parAxis}{\targmii{b}{\comone{smca}{a}}}
}{% Example Function Call
    \parAxis{b}{\smca{a}}
}
\index{equations!inertia|)}

\newpage

\subsection*{\blu{Short Inertia Equations}}
\label{sec:inereqssh}
\index{equations!inertia!shorthand|(}

\entryl{% Command Name
    rRelsh
}{% Arguments
    \targsiii{frame}{sys}{point}
}{% Description
    Relative displacement shorthand. This is the shorthand relative displacement vector for the inertia tensor of the system \meta{sys} relative to the point \meta{point} with respect to the frame \meta{frame}. 
}{% Argument Descriptions
    \inputsiii{%
    	frame}{Frame; letter (a-z, A-Z)}{%
    	sys}{System name; expression}{%
    	point}{Reference point; expression}%
}{% Example Description
    Say we want to define the relative displacement using shorthand summation of a mass element $m_i$ relative to the point $B$ in the body frame $\fr{B}$
}{% Example Text Arguments
    \comarg{rRelsh}{\targmiii{b}{m\string_i}{\comone{smca}{B}}}
}{% Example Function Call
    \rRelsh{b}{m_i}{\smca{B}}
}

\entryl{% Command Name
    IxxSumsh
}{% Arguments
    { }
}{% Description
    $I_{xx}$ sum shorthand. Defines the $I_{xx}$ term of the inertia tensor using summation and shorthand notation.
}{% Argument Descriptions
    \textit{No input arguments.}
}{% Example Description
    Define the $I_{xx}$ term of the inertia tensor using shorthand summation notation.
}{% Example Text Arguments
    \cs{IxxSumsh}
}{% Example Function Call
    \IxxSumsh
}

\newpage

\entryl{% Command Name
    IxySumsh
}{% Arguments
    { }
}{% Description
    $I_{xy}$ sum shorthand. Defines the $I_{xy}$ term of the inertia tensor using summation and shorthand notation.
}{% Argument Descriptions
    \textit{No input arguments.}
}{% Example Description
    Define the $I_{xy}$ term of the inertia tensor using shorthand summation notation.
}{% Example Text Arguments
    \cs{IxySumsh}
}{% Example Function Call
    \IxySumsh
}

\entryl{% Command Name
    IxzSumsh
}{% Arguments
    { }
}{% Description
    $I_{xz}$ sum shorthand. Defines the $I_{xz}$ term of the inertia tensor using summation and shorthand notation.
}{% Argument Descriptions
    \textit{No input arguments.}
}{% Example Description
    Define the $I_{xz}$ term of the inertia tensor using shorthand summation notation.
}{% Example Text Arguments
    \cs{IxzSumsh}
}{% Example Function Call
    \IxzSumsh
}

\newpage

\entryl{% Command Name
    IyxSumsh
}{% Arguments
    { }
}{% Description
    $I_{yx}$ sum shorthand. Defines the $I_{yx}$ term of the inertia tensor using summation and shorthand notation.
}{% Argument Descriptions
    \textit{No input arguments.}
}{% Example Description
    Define the $I_{yx}$ term of the inertia tensor using shorthand summation notation.
}{% Example Text Arguments
    \cs{IyxSumsh}
}{% Example Function Call
    \IyxSumsh
}

\entryl{% Command Name
    IyySumsh
}{% Arguments
    { }
}{% Description
    $I_{yy}$ sum shorthand. Defines the $I_{yy}$ term of the inertia tensor using summation and shorthand notation.
}{% Argument Descriptions
    \textit{No input arguments.}
}{% Example Description
    Define the $I_{yy}$ term of the inertia tensor using shorthand summation notation.
}{% Example Text Arguments
    \cs{IyySumsh}
}{% Example Function Call
    \IyySumsh
}

\newpage

\entryl{% Command Name
    IyzSumsh
}{% Arguments
    { }
}{% Description
    $I_{yz}$ sum shorthand. Defines the $I_{yz}$ term of the inertia tensor using summation and shorthand notation.
}{% Argument Descriptions
    \textit{No input arguments.}
}{% Example Description
    Define the $I_{yz}$ term of the inertia tensor using shorthand summation notation.
}{% Example Text Arguments
    \cs{IyzSumsh}
}{% Example Function Call
    \IyzSumsh
}

\entryl{% Command Name
    IzxSumsh
}{% Arguments
    { }
}{% Description
    $I_{zx}$ sum shorthand. Defines the $I_{zx}$ term of the inertia tensor using summation and shorthand notation.
}{% Argument Descriptions
    \textit{No input arguments.}
}{% Example Description
    Define the $I_{zx}$ term of the inertia tensor using shorthand summation notation.
}{% Example Text Arguments
    \cs{IzxSumsh}
}{% Example Function Call
    \IzxSumsh
}

\newpage

\entryl{% Command Name
    IzySumsh
}{% Arguments
    { }
}{% Description
    $I_{zy}$ sum shorthand. Defines the $I_{zy}$ term of the inertia tensor using summation and shorthand notation.
}{% Argument Descriptions
    \textit{No input arguments.}
}{% Example Description
    Define the $I_{zy}$ term of the inertia tensor using shorthand summation notation.
}{% Example Text Arguments
    \cs{IzySumsh}
}{% Example Function Call
    \IzySumsh
}

\entryl{% Command Name
    IzzSumsh
}{% Arguments
    { }
}{% Description
    $I_{zz}$ sum shorthand. Defines the $I_{zz}$ term of the inertia tensor using summation and shorthand notation.
}{% Argument Descriptions
    \textit{No input arguments.}
}{% Example Description
    Define the $I_{zz}$ term of the inertia tensor using shorthand summation notation.
}{% Example Text Arguments
    \cs{IzzSumsh}
}{% Example Function Call
    \IzzSumsh
}

\newpage

\entryl{% Command Name
    IxxInt
}{% Arguments
    { }
}{% Description
    $I_{xx}$ sum shorthand. Defines the $I_{xx}$ term of the inertia tensor using integration and shorthand notation.
}{% Argument Descriptions
    \textit{No input arguments.}
}{% Example Description
    Define the $I_{xx}$ term of the inertia tensor using shorthand integration notation.
}{% Example Text Arguments
    \cs{IxxInt}
}{% Example Function Call
    \IxxInt
}

\entryl{% Command Name
    IxyInt
}{% Arguments
    { }
}{% Description
    $I_{xy}$ sum shorthand. Defines the $I_{xy}$ term of the inertia tensor using integration and shorthand notation.
}{% Argument Descriptions
    \textit{No input arguments.}
}{% Example Description
    Define the $I_{xy}$ term of the inertia tensor using shorthand integration notation.
}{% Example Text Arguments
    \cs{IxyInt}
}{% Example Function Call
    \IxyInt
}

\newpage

\entryl{% Command Name
    IxzInt
}{% Arguments
    { }
}{% Description
    $I_{xz}$ sum shorthand. Defines the $I_{xz}$ term of the inertia tensor using integration and shorthand notation.
}{% Argument Descriptions
    \textit{No input arguments.}
}{% Example Description
    Define the $I_{xz}$ term of the inertia tensor using shorthand integration notation.
}{% Example Text Arguments
    \cs{IxzInt}
}{% Example Function Call
    \IxzInt
}

\entryl{% Command Name
    IyxInt
}{% Arguments
    { }
}{% Description
    $I_{yx}$ sum shorthand. Defines the $I_{yx}$ term of the inertia tensor using integration and shorthand notation.
}{% Argument Descriptions
    \textit{No input arguments.}
}{% Example Description
    Define the $I_{yx}$ term of the inertia tensor using shorthand integration notation.
}{% Example Text Arguments
    \cs{IyxInt}
}{% Example Function Call
    \IyxInt
}

\newpage

\entryl{% Command Name
    IyyInt
}{% Arguments
    { }
}{% Description
    $I_{yy}$ sum shorthand. Defines the $I_{yy}$ term of the inertia tensor using integration and shorthand notation.
}{% Argument Descriptions
    \textit{No input arguments.}
}{% Example Description
    Define the $I_{yy}$ term of the inertia tensor using shorthand integration notation.
}{% Example Text Arguments
    \cs{IyyInt}
}{% Example Function Call
    \IyyInt
}

\entryl{% Command Name
    IyzInt
}{% Arguments
    { }
}{% Description
    $I_{yz}$ sum shorthand. Defines the $I_{yz}$ term of the inertia tensor using integration and shorthand notation.
}{% Argument Descriptions
    \textit{No input arguments.}
}{% Example Description
    Define the $I_{yz}$ term of the inertia tensor using shorthand integration notation.
}{% Example Text Arguments
    \cs{IyzInt}
}{% Example Function Call
    \IyzInt
}

\newpage

\entryl{% Command Name
    IzxInt
}{% Arguments
    { }
}{% Description
    $I_{zx}$ sum shorthand. Defines the $I_{zx}$ term of the inertia tensor using integration and shorthand notation.
}{% Argument Descriptions
    \textit{No input arguments.}
}{% Example Description
    Define the $I_{zx}$ term of the inertia tensor using shorthand integration notation.
}{% Example Text Arguments
    \cs{IzxInt}
}{% Example Function Call
    \IzxInt
}

\entryl{% Command Name
    IzyInt
}{% Arguments
    { }
}{% Description
    $I_{zy}$ sum shorthand. Defines the $I_{zy}$ term of the inertia tensor using integration and shorthand notation.
}{% Argument Descriptions
    \textit{No input arguments.}
}{% Example Description
    Define the $I_{zy}$ term of the inertia tensor using shorthand integration notation.
}{% Example Text Arguments
    \cs{IzyInt}
}{% Example Function Call
    \IzyInt
}

\newpage

\entryl{% Command Name
    IzzInt
}{% Arguments
    { }
}{% Description
    $I_{zz}$ sum shorthand. Defines the $I_{zz}$ term of the inertia tensor using integration and shorthand notation.
}{% Argument Descriptions
    \textit{No input arguments.}
}{% Example Description
    Define the $I_{zz}$ term of the inertia tensor using shorthand integration notation.
}{% Example Text Arguments
    \cs{IzzInt}
}{% Example Function Call
    \IzzInt
}

\index{theorem!parallel axis!shorthand}
\entryl{% Command Name
    parAxissh
}{% Arguments
    { }
}{% Description
    Parallel axis theorem in shorthand matrix form. Defines the terms of the inertia tensor about using the parallel axis theorem and shorthand.
}{% Argument Descriptions
    \textit{No input arguments.}
}{% Example Description
    Say we want to define the inertia tensor in matrix form using the parallel axis theorem in shorthand notation.
}{% Example Text Arguments
    \cs{parAxissh}
}{% Example Function Call
    \parAxissh
}
\index{equations!inertia!shorthand|)}

\newpage

\subsection*{\blu{Angular Velocity Equations}}
\label{sec:angveleqs}
\index{equations!angular velocity|(}

\entryd{% Command Name
    angVdef
}{% Arguments
    \targsii{frame1}{frame2}
}{% Description
    Angular velocity definition. Defines the angular velocity of frame \meta{frame2} relative to frame \meta{frame1}.
}{% Argument Descriptions
    \inputsii{%
    	frame1}{Frame; letter (a-z, A-Z)}{%
    	frame2}{Frame; letter (a-z, A-Z)}%
}{% Example Description
    Say we want to define the angular velocity of the $\fr{b}$ frame relative to the $\fr{a}$ frame.
}{% Example Text Arguments
    \comarg{angVdef}{\targmii{a}{b}}
}{% Example Function Call
    \angVdef{a}{b}
}

\entryd{% Command Name
    angVxdef
}{% Arguments
    \targsii{frame1}{frame2}
}{% Description
    Angular velocity x-component definition. Defines the x-component of the angular velocity of frame \meta{frame2} relative to frame \meta{frame1}.
}{% Argument Descriptions
    \inputsii{%
    	frame1}{Frame; letter (a-z, A-Z)}{%
    	frame2}{Frame; letter (a-z, A-Z)}%
}{% Example Description
    Say we want to define the x-component of the angular velocity of the $\fr{b}$ frame relative to the $\fr{a}$ frame.
}{% Example Text Arguments
    \comarg{angVxdef}{\targmii{a}{b}}
}{% Example Function Call
    \angVxdef{a}{b}
}

\newpage

\entryd{% Command Name
    angVydef
}{% Arguments
    \targsii{frame1}{frame2}
}{% Description
    Angular velocity y-component definition. Defines the y-component of the angular velocity of frame \meta{frame2} relative to frame \meta{frame1}.
}{% Argument Descriptions
    \inputsii{%
    	frame1}{Frame; letter (a-z, A-Z)}{%
    	frame2}{Frame; letter (a-z, A-Z)}%
}{% Example Description
    Say we want to define the y-component of the angular velocity of the $\fr{b}$ frame relative to the $\fr{a}$ frame.
}{% Example Text Arguments
    \comarg{angVydef}{\targmii{a}{b}}
}{% Example Function Call
    \angVydef{a}{b}
}

\entryd{% Command Name
    angVzdef
}{% Arguments
    \targsii{frame1}{frame2}
}{% Description
    Angular velocity z-component definition. Defines the z-component of the angular velocity of frame \meta{frame2} relative to frame \meta{frame1}.
}{% Argument Descriptions
    \inputsii{%
    	frame1}{Frame; letter (a-z, A-Z)}{%
    	frame2}{Frame; letter (a-z, A-Z)}%
}{% Example Description
    Say we want to define the z-component of the angular velocity of the $\fr{b}$ frame relative to the $\fr{a}$ frame.
}{% Example Text Arguments
    \comarg{angVzdef}{\targmii{a}{b}}
}{% Example Function Call
    \angVzdef{a}{b}
}

\newpage

\entryd{% Command Name
    Pvecdef
}{% Arguments
    \targsiii{frame1}{frame2}{var}
}{% Description
    P-vector definition. Defines the P-vector of frame \meta{frame2} relative to frame \meta{frame1} with respect to \meta{var}.
}{% Argument Descriptions
    \inputsiii{%
    	frame1}{Frame; letter (a-z, A-Z)}{%
    	frame2}{Frame; letter (a-z, A-Z)}{%
    	var}{Variable; symbol}%
}{% Example Description
    Say we want to define the P-vector of the $\fr{b}$ frame relative to the $\fr{a}$ frame with respect to $\theta$.
}{% Example Text Arguments
    \comarg{Pvecdef}{\targmiii{a}{b}{\cs{theta}}}
}{% Example Function Call
    \Pvecdef{a}{b}{\theta}
}

\entryd{% Command Name
    Pvecxdef
}{% Arguments
    \targsiii{frame1}{frame2}{var}
}{% Description
    P-vector x-component definition. Defines the x-component of the P-vector of frame \meta{frame2} relative to frame \meta{frame1} with respect to \meta{var}.
}{% Argument Descriptions
    \inputsiii{%
    	frame1}{Frame; letter (a-z, A-Z)}{%
    	frame2}{Frame; letter (a-z, A-Z)}{%
    	var}{Variable; symbol}%
}{% Example Description
    Say we want to define the x-component of the P-vector of the $\fr{b}$ frame relative to the $\fr{a}$ frame with respect to $\theta$.
}{% Example Text Arguments
    \comarg{Pvecxdef}{\targmiii{a}{b}{\cs{theta}}}
}{% Example Function Call
    \Pvecxdef{a}{b}{\theta}
}

\newpage

\entryd{% Command Name
    Pvecydef
}{% Arguments
    \targsiii{frame1}{frame2}{var}
}{% Description
    P-vector y-component definition. Defines the y-component of the P-vector of frame \meta{frame2} relative to frame \meta{frame1} with respect to \meta{var}.
}{% Argument Descriptions
    \inputsiii{%
    	frame1}{Frame; letter (a-z, A-Z)}{%
    	frame2}{Frame; letter (a-z, A-Z)}{%
    	var}{Variable; symbol}%
}{% Example Description
    Say we want to define the y-component of the P-vector of the $\fr{b}$ frame relative to the $\fr{a}$ frame with respect to $\theta$.
}{% Example Text Arguments
    \comarg{Pvecydef}{\targmiii{a}{b}{\cs{theta}}}
}{% Example Function Call
    \Pvecydef{a}{b}{\theta}
}

\entryd{% Command Name
    Pveczdef
}{% Arguments
    \targsiii{frame1}{frame2}{var}
}{% Description
    P-vector z-component definition. Defines the z-component of the P-vector of frame \meta{frame2} relative to frame \meta{frame1} with respect to \meta{var}.
}{% Argument Descriptions
    \inputsiii{%
    	frame1}{Frame; letter (a-z, A-Z)}{%
    	frame2}{Frame; letter (a-z, A-Z)}{%
    	var}{Variable; symbol}%
}{% Example Description
    Say we want to define the z-component of the P-vector of the $\fr{b}$ frame relative to the $\fr{a}$ frame with respect to $\theta$.
}{% Example Text Arguments
    \comarg{Pveczdef}{\targmiii{a}{b}{\cs{theta}}}
}{% Example Function Call
    \Pveczdef{a}{b}{\theta}
}

\newpage

\entryd{% Command Name
    Tvecdef
}{% Arguments
    \targsiii{frame1}{frame2}{var}
}{% Description
    T-vector definition. Defines the T-vector of frame \meta{frame2} relative to frame \meta{frame1} with respect to \meta{var}.
}{% Argument Descriptions
    \inputsiii{%
    	frame1}{Frame; letter (a-z, A-Z)}{%
    	frame2}{Frame; letter (a-z, A-Z)}{%
    	var}{Variable; symbol}%
}{% Example Description
    Say we want to define the T-vector of the $\fr{b}$ frame relative to the $\fr{a}$ frame with respect to $\beta$.
}{% Example Text Arguments
    \comarg{Tvecdef}{\targmiii{a}{b}{\cs{beta}}}
}{% Example Function Call
    \Tvecdef{a}{b}{\beta}
}

\entryd{% Command Name
    Tvecxdef
}{% Arguments
    \targsiii{frame1}{frame2}{var}
}{% Description
    T-vector x-component definition. Defines the x-component of the T-vector of frame \meta{frame2} relative to frame \meta{frame1} with respect to \meta{var}.
}{% Argument Descriptions
    \inputsiii{%
    	frame1}{Frame; letter (a-z, A-Z)}{%
    	frame2}{Frame; letter (a-z, A-Z)}{%
    	var}{Variable; symbol}%
}{% Example Description
    Say we want to define the x-component of the T-vector of the $\fr{b}$ frame relative to the $\fr{a}$ frame with respect to $\beta$.
}{% Example Text Arguments
    \comarg{Tvecxdef}{\targmiii{a}{b}{\cs{beta}}}
}{% Example Function Call
    \Tvecxdef{a}{b}{\beta}
}

\newpage

\entryd{% Command Name
    Tvecydef
}{% Arguments
    \targsiii{frame1}{frame2}{var}
}{% Description
    T-vector y-component definition. Defines the y-component of the T-vector of frame \meta{frame2} relative to frame \meta{frame1} with respect to \meta{var}.
}{% Argument Descriptions
    \inputsiii{%
    	frame1}{Frame; letter (a-z, A-Z)}{%
    	frame2}{Frame; letter (a-z, A-Z)}{%
    	var}{Variable; symbol}%
}{% Example Description
    Say we want to define the y-component of the T-vector of the $\fr{b}$ frame relative to the $\fr{a}$ frame with respect to $\beta$.
}{% Example Text Arguments
    \comarg{Tvecydef}{\targmiii{a}{b}{\cs{beta}}}
}{% Example Function Call
    \Tvecydef{a}{b}{\beta}
}

\entryd{% Command Name
    Tveczdef
}{% Arguments
    \targsiii{frame1}{frame2}{var}
}{% Description
    T-vector z-component definition. Defines the z-component of the T-vector of frame \meta{frame2} relative to frame \meta{frame1} with respect to \meta{var}.
}{% Argument Descriptions
    \inputsiii{%
    	frame1}{Frame; letter (a-z, A-Z)}{%
    	frame2}{Frame; letter (a-z, A-Z)}{%
    	var}{Variable; symbol}%
}{% Example Description
    Say we want to define the z-component of the T-vector of the $\fr{b}$ frame relative to the $\fr{a}$ frame with respect to $\beta$.
}{% Example Text Arguments
    \comarg{Tveczdef}{\targmiii{a}{b}{\cs{beta}}}
}{% Example Function Call
    \Tveczdef{a}{b}{\beta}
}
\index{equations!angular velocity|)}

\newpage

\subsection*{\blu{Angular Momentum Equations}}
\label{sec:angmomeqs}
\index{equations!angular momentum|(}

\entryd{% Command Name
    angMomPart
}{% Arguments
    \targsiv{frame}{point1}{point2}{mass}
}{% Description
    Particle angular momentum definition. This command is used to define the angular momentum of a particle with mass \meta{mass} with respect to \meta{point2} with respect to \meta{point1}, relative to the frame \meta{frame}.
}{% Argument Descriptions
    \inputsiv{%
        frame}{Letter (a-z, A-Z)}{%
        point1}{Name of point1; expression}{%
        point2}{Name of point2; expression}{%
        mass}{Name of mass; expression}%
}{% Example Description
    Say we want to define the angular momentum of a particle of mass $m$ with respect to $P$ with respect to $Q$ expressed in the $\fr{f}$ frame.
}{% Example Text Arguments
    \comarg{angMomPart}{\targmiv{f}{\comone{smca}{q}}{\comone{smca}{p}}{m}}
}{% Example Function Call \comone{smca}{}
    \angMomPart{f}{\smca{q}}{\smca{p}}{m}
}

\newpage

\entryl{% Command Name
    angMomSys
}{% Arguments
    \targsiii{frame}{point1}{point2}
}{% Description
    System angular momentum. This command is used to express the angular momentum of a system with respect to \meta{point2} with respect to \meta{point1}, relative to the frame \meta{frame}.
}{% Argument Descriptions
    \inputsiii{%
        frame}{Letter (a-z, A-Z)}{%
        point1}{Name of point1; expression}{%
        point2}{Name of point2; expression}%
}{% Example Description
    Say we want to define the angular momentum of a system with respect to $P$ with respect to $Q$ expressed in the $\fr{F}$ frame.
}{% Example Text Arguments
    \comarg{angMomSys}{\targmiii{f}{\comone{smca}{Q}}{\comone{smca}{P}}}
}{% Example Function Call
    \angMomSys{f}{\smca{Q}}{\smca{P}}
}

\entryd{% Command Name
    angMomDefI
}{% Arguments
    { }
}{% Description
    Angular momentum definition I. Generic angular momentum expression.
}{% Argument Descriptions
    \textit{No input arguments.}
}{% Example Description
    Show the generic expression for angular momentum.
}{% Example Text Arguments
    \cs{angMomDefI}
}{% Example Function Call
    \angMomDefI
}

\newpage

\entryd{% Command Name
    angMomDefII
}{% Arguments
    { }
}{% Description
    Angular momentum definition II. Angular momentum of a rigid body.
}{% Argument Descriptions
    \textit{No input arguments.}
}{% Example Description
    Show the expression for the angular momentum of a rigid body.
}{% Example Text Arguments
    \cs{angMomDefII}
}{% Example Function Call
    \angMomDefII
}

\entryd{% Command Name
    angMomDefIII
}{% Arguments
    { }
}{% Description
    Angular momentum definition III. Angular momentum about the origin with respect to a body frame.
}{% Argument Descriptions
    \textit{No input arguments.}
}{% Example Description
    Show the angular momentum definition III.
}{% Example Text Arguments
    \cs{angMomDefIII}
}{% Example Function Call
    \angMomDefIII
}

\newpage

\entryd{% Command Name
    angMomDefIV
}{% Arguments
    { }
}{% Description
    Angular momentum definition IV. Angular momentum about an arbitrary point $A$.
}{% Argument Descriptions
    \textit{No input arguments.}
}{% Example Description
    Show the angular momentum definition IV.
}{% Example Text Arguments
    \cs{angMomDefIV}
}{% Example Function Call
    \angMomDefIV
}
\index{equations!angular momentum|)}

\subsection*{\blu{Torque Equations}}
\label{sec:torqueeqs}
\index{equations!torque|(}

\entryl{% Command Name
    torqueDef
}{% Arguments
    { }
}{% Description
    Torque definition. Torque definition for a system of $n$ particles about point $P$.
}{% Argument Descriptions
    \textit{No input arguments.}
}{% Example Description
    Show the torque definition.
}{% Example Text Arguments
    \cs{torqueDef}
}{% Example Function Call
    \torqueDef
}

\newpage

\entryd{% Command Name
    torqueDefI
}{% Arguments
    { }
}{% Description
    Torque definition I. Most general case, $\fr{O}$ is an IRF.
}{% Argument Descriptions
    \textit{No input arguments.}
}{% Example Description
    Show torque definition I.
}{% Example Text Arguments
    \cs{torqueDefI}
}{% Example Function Call
    \torqueDefI
}

\entryd{% Command Name
    torqueDefII
}{% Arguments
    { }
}{% Description
    Torque definition II. $Q$ = $O$, $P$ = $\mathit{CM}$, $\fr{O}$ is an IRF.
}{% Argument Descriptions
    \textit{No input arguments.}
}{% Example Description
    Show torque definition II.
}{% Example Text Arguments
    \cs{torqueDefII}
}{% Example Function Call
    \torqueDefII
}

\newpage

\entryd{% Command Name
    torqueDefIII
}{% Arguments
    { }
}{% Description
    Torque definition III. $Q$ = $P$ = $\mathit{CM}$, $\fr{O}$ is an IRF.
}{% Argument Descriptions
    \textit{No input arguments.}
}{% Example Description
    Show torque definition III.
}{% Example Text Arguments
    \cs{torqueDefIII}
}{% Example Function Call
    \torqueDefIII
}

\entryd{% Command Name
    torqueDefIV
}{% Arguments
    { }
}{% Description
    Torque definition IV. $Q$ = $P$ = $O$, $\fr{O}$ is an IRF.
}{% Argument Descriptions
    \textit{No input arguments.}
}{% Example Description
    Show torque definition IV.
}{% Example Text Arguments
    \cs{torqueDefIV}
}{% Example Function Call
    \torqueDefIV
}

\newpage

\entryd{% Command Name
    torqueDefV
}{% Arguments
    { }
}{% Description
    Torque definition V. $Q$ = $O$, $P$ is a fixed point with respect to $O$, $\fr{O}$ is an IRF.
}{% Argument Descriptions
    \textit{No input arguments.}
}{% Example Description
    Show torque definition V.
}{% Example Text Arguments
    \cs{torqueDefV}
}{% Example Function Call
    \torqueDefV
}
\index{equations!torque|)}

\subsection*{\blu{Kinetic Energy Equations}}
\label{sec:kineneqs}
\index{equations!kinetic energy|(}

\entryl{% Command Name
    kinEnDef
}{% Arguments
    { }
}{% Description
    Kinetic energy definition. Kinetic energy definition for a system of $n$ particles.
}{% Argument Descriptions
    \textit{No input arguments.}
}{% Example Description
    Show the kinetic energy definition.
}{% Example Text Arguments
    \cs{kinEnDef}
}{% Example Function Call
    \kinEnDef
}

\newpage

\entryl{% Command Name
    kinEnDefI
}{% Arguments
    { }
}{% Description
    Kinetic energy definition I. General expression for the kinetic energy for a system of $n$ particles and any frames $\fr{O}$ and $\fr{A}$.
}{% Argument Descriptions
    \textit{No input arguments.}
}{% Example Description
    Show kinetic energy definition I.
}{% Example Text Arguments
    \cs{kinEnDefI}
}{% Example Function Call
    \kinEnDefI
}

\entryd{% Command Name
    kinEnDefII
}{% Arguments
    { }
}{% Description
    Kinetic energy definition II. Expression for the kinetic energy for a rigid body with body frame $\fr{A}$. Valid for any frames $\fr{O}$ and $\fr{A}$.
}{% Argument Descriptions
    \textit{No input arguments.}
}{% Example Description
    Show kinetic energy definition II.
}{% Example Text Arguments
    \cs{kinEnDefII}
}{% Example Function Call
    \kinEnDefII
}

\newpage

\entryd{% Command Name
    kinEnDefIII
}{% Arguments
    { }
}{% Description
    Kinetic energy definition III. Expression for the kinetic energy for a rigid body with body frame $\fr{A}$ and $A$ = $\mathit{CM}$. Valid for any frames $\fr{O}$ and $\fr{A}$.
}{% Argument Descriptions
    \textit{No input arguments.}
}{% Example Description
    Show kinetic energy definition III.
}{% Example Text Arguments
    \cs{kinEnDefIII}
}{% Example Function Call
    \kinEnDefIII
}

\entryd{% Command Name
    kinEnDefIV
}{% Arguments
    { }
}{% Description
    Kinetic energy definition IV. Expression for the kinetic energy for a rigid body with body frame $\fr{A}$ and $A$ is a fixed point with respect to $O$ ($\relV{o}{\smca{A}}{\smca{O}}$ = $\vec{0}\,$). Valid for any frames $\fr{O}$ and $\fr{A}$.
}{% Argument Descriptions
    \textit{No input arguments.}
}{% Example Description
    Show kinetic energy definition IV.
}{% Example Text Arguments
    \cs{kinEnDefIV}
}{% Example Function Call
    \kinEnDefIV
}
\index{equations!kinetic energy|)}

\newpage

\subsection*{\blu{Alternate Kinetic Energy Equations}}
\label{sec:altkineneqs}
\index{equations!kinetic energy!alternate|(}
\index{Liapunov|see {kinetic energy, alternate}}
\index{Ljapunov|see {kinetic energy, alternate}}
\index{stability|see {kinetic energy, alternate}}

\entryd{% Command Name
    TEq
}{% Arguments
    { }
}{% Description
    Alternative kinetic energy definition. Used to evaluate Ljapunov/Liapunov stability.
}{% Argument Descriptions
    \textit{No input arguments.}
}{% Example Description
    Show the alternative kinetic energy definition.
}{% Example Text Arguments
    \cs{TEq}
}{% Example Function Call
    \TEq
}

\entryd{% Command Name
    TzeroEq
}{% Arguments
    { }
}{% Description
    $T_0$ equation. This command defines the zeroth-order term of the kinetic energy.
}{% Argument Descriptions
    \textit{No input arguments.}
}{% Example Description
    Show the definition of $T_0$.
}{% Example Text Arguments
    \cs{TzeroEq}
}{% Example Function Call
    \TzeroEq
}

\newpage

\entryl{% Command Name
    ToneEq
}{% Arguments
    { }
}{% Description
    $T_1$ equation. This command defines the first-order term of the kinetic energy.
}{% Argument Descriptions
    \textit{No input arguments.}
}{% Example Description
    Show the definition of $T_1$.
}{% Example Text Arguments
    \cs{ToneEq}
}{% Example Function Call
    \ToneEq
}

\entryl{% Command Name
    TtwoEq
}{% Arguments
    { }
}{% Description
    $T_2$ equation. This command defines the second-order term of the kinetic energy.
}{% Argument Descriptions
    \textit{No input arguments.}
}{% Example Description
    Show the definition of $T_2$.
}{% Example Text Arguments
    \cs{TtwoEq}
}{% Example Function Call
    \TtwoEq
}
\index{equations!kinetic energy!alternate|)}

\newpage

\subsection*{\blu{Euler's Equations of Motion}}
\label{sec:eulerseqs}
\index{equations!of motion!Euler's|(}
\index{Euler|see {equations, of motion, Euler's}}

\entryd{% Command Name
    EulerEqx
}{% Arguments
    { }
}{% Description
    Euler's equation about the x-axis. This command defines Euler's equation of motion about the x-axis for a rigid body about its principal axes.
}{% Argument Descriptions
    \textit{No input arguments.}
}{% Example Description
    Show the definition of Euler's equation of motion about the x-axis.
}{% Example Text Arguments
    \cs{EulerEqx}
}{% Example Function Call
    \EulerEqx
}

\entryd{% Command Name
    EulerEqy
}{% Arguments
    { }
}{% Description
    Euler's equation about the y-axis. This command defines Euler's equation of motion about the y-axis for a rigid body about its principal axes.
}{% Argument Descriptions
    \textit{No input arguments.}
}{% Example Description
    Show the definition of Euler's equation of motion about the y-axis.
}{% Example Text Arguments
    \cs{EulerEqy}
}{% Example Function Call
    \EulerEqy
}

\newpage

\entryd{% Command Name
    EulerEqz
}{% Arguments
    { }
}{% Description
    Euler's equation about the z-axis. This command defines Euler's equation of motion about the z-axis for a rigid body about its principal axes.
}{% Argument Descriptions
    \textit{No input arguments.}
}{% Example Description
    Show the definition of Euler's equation of motion about the z-axis.
}{% Example Text Arguments
    \cs{EulerEqz}
}{% Example Function Call
    \EulerEqz
}
\index{equations!of motion!Euler's|)}

\subsection*{\blu{Lagrange's Equations}}
\label{sec:lagrangeseqs}
\index{equations!Lagrange's|see {Lagrange's equations}}
\index{Lagrange's equations|(}

\entryd{% Command Name
    Lagrangian
}{% Arguments
    { }
}{% Description
    Lagrangian. This command defines the Lagrangian. You can use \code{\$\comone{mathcal}{L}\$} to get $\mathcal{L}$.
}{% Argument Descriptions
    \textit{No input arguments.}
}{% Example Description
    Show the definition of the Lagrangian.
}{% Example Text Arguments
    \cs{Lagrangian}
}{% Example Function Call
    \Lagrangian
}

\newpage

\entryl{% Command Name
    Lagrange
}{% Arguments
    \targsi{num}
}{% Description
    Lagrange's equation. This command defines Lagrange's equation of a specified $q_n$ where $n$ = \meta{num}.
}{% Argument Descriptions
    \inputsi{num}{Number; positive integer}
}{% Example Description
    Say we want to define Lagrange's equation for an arbitrary $q_n$.
}{% Example Text Arguments
    \comarg{Lagrange}{\targmi{n}}
}{% Example Function Call
    \Lagrange{n}
}

\entryl{% Command Name
    LagrangeTV
}{% Arguments
    \targsi{num}
}{% Description
    Lagrange's equation energy form. This command defines Lagrange's equation of a specified $q_n$ where $n$ = \meta{num} in terms of the kinetic and potential energy.
}{% Argument Descriptions
    \inputsi{num}{Number; positive integer}
}{% Example Description
    Say we want to define Lagrange's equation in energy form for an arbitrary $q_n$.
}{% Example Text Arguments
    \comarg{LagrangeTV}{\targmi{n}}
}{% Example Function Call
    \LagrangeTV{n}
}
\index{Lagrange's equations|)}

\newpage

\subsection*{\blu{Kane's Equations}}
\label{sec:kaneseqs}
\index{equations!Kane's|see {Kane's equations}}
\index{Kane's equations|(}

\entryd{% Command Name
    KaneEq
}{% Arguments
    { }
}{% Description
    Kane's equation. This command defines Kane's equation of motion.
}{% Argument Descriptions
    \textit{No input arguments.}
}{% Example Description
    Display Kane's equation of motion.
}{% Example Text Arguments
    \cs{KaneEq}
}{% Example Function Call
    \KaneEq
}

\entryl{% Command Name
    Kaneqdot
}{% Arguments
    { }
}{% Description
    Kane's $\dot{q}_{s}$ definition. Defines the time derivative of the generalized coordinate.
}{% Argument Descriptions
    \textit{No input arguments.}
}{% Example Description
    Display Kane's $\dot{q}_{s}$ definition.
}{% Example Text Arguments
    \cs{Kaneqdot}
}{% Example Function Call
    \Kaneqdot
}

\newpage

\entryl{% Command Name
    Kaneomegar
}{% Arguments
    { }
}{% Description
    Kane's $\pangVr{O}{B}{r}$ definition. Defines the rth partial angular velocity.
}{% Argument Descriptions
    \textit{No input arguments.}
}{% Example Description
    Display Kane's $\pangVr{O}{B}{r}$ definition.
}{% Example Text Arguments
    \cs{Kaneomegar}
}{% Example Function Call
    \Kaneomegar
}

\entryl{% Command Name
    Kaneomegat
}{% Arguments
    { }
}{% Description
    Kane's $\pangVt{O}{B}$ definition. Defines the time partial angular velocity.
}{% Argument Descriptions
    \textit{No input arguments.}
}{% Example Description
    Display Kane's $\pangVt{O}{B}$ definition.
}{% Example Text Arguments
    \cs{Kaneomegat}
}{% Example Function Call
    \Kaneomegat
}

\newpage

\entryl{% Command Name
    Kaneomega
}{% Arguments
    { }
}{% Description
    Kane's $\angV{O}{B}$ definition. Defines the angular velocity.
}{% Argument Descriptions
    \textit{No input arguments.}
}{% Example Description
    Display Kane's $\angV{O}{B}$ definition.
}{% Example Text Arguments
    \cs{Kaneomega}
}{% Example Function Call
    \Kaneomega
}

\entryl{% Command Name
    Kanevcmr
}{% Arguments
    { }
}{% Description
    Kane's $\ptraVr{O}{\CM}{r}$ definition. Defines the rth partial velocity.
}{% Argument Descriptions
    \textit{No input arguments.}
}{% Example Description
    Display Kane's $\ptraVr{O}{\CM}{r}$ definition.
}{% Example Text Arguments
    \cs{Kanevcmr}
}{% Example Function Call
    \Kanevcmr
}

\newpage

\entryl{% Command Name
    Kanevcmt
}{% Arguments
    { }
}{% Description
    Kane's $\ptraVr{O}{\CM}{t}$ definition. Defines the time partial velocity.
}{% Argument Descriptions
    \textit{No input arguments.}
}{% Example Description
    Display Kane's $\ptraVr{O}{\CM}{t}$ definition.
}{% Example Text Arguments
    \cs{Kanevcmt}
}{% Example Function Call
    \Kanevcmt
}

\entryl{% Command Name
    Kanevcm
}{% Arguments
    { }
}{% Description
    Kane's $\relV{O}{\CM}{\smca{O}}$ definition. Defines the velocity.
}{% Argument Descriptions
    \textit{No input arguments.}
}{% Example Description
    Display Kane's $\relV{O}{\CM}{\smca{O}}$ definition.
}{% Example Text Arguments
    \cs{Kanevcm}
}{% Example Function Call
    \Kanevcm
}

\newpage

\entryl{% Command Name
    KaneFrPart
}{% Arguments
    { }
}{% Description
    Kane's $\KFr{r}$ term for particles.
}{% Argument Descriptions
    \textit{No input arguments.}
}{% Example Description
    Show Kane's $\KFr{r}$ term for particles.
}{% Example Text Arguments
    \cs{KaneFrPart}
}{% Example Function Call
    \KaneFrPart
}

\newpage

\entryl{% Command Name
    KaneFrsPart
}{% Arguments
    { }
}{% Description
    Kane's $\KFrs{r}$ term for particles.
}{% Argument Descriptions
    \textit{No input arguments.}
}{% Example Description
    Show Kane's $\KFrs{r}$ term for particles.
}{% Example Text Arguments
    \cs{KaneFrsPart}
}{% Example Function Call
    \KaneFrsPart
}

\entryl{% Command Name
    KaneFrRig
}{% Arguments
    { }
}{% Description
    Kane's $\KFr{r}$ term for rigid bodies.
}{% Argument Descriptions
    \textit{No input arguments.}
}{% Example Description
    Show Kane's $\KFr{r}$ term for rigid bodies.
}{% Example Text Arguments
    \cs{KaneFrRig}
}{% Example Function Call
    \KaneFrRig
}

\newpage

\entryl{% Command Name
    KaneFrsRig
}{% Arguments
    { }
}{% Description
    Kane's $\KFrs{r}$ term for rigid bodies.
}{% Argument Descriptions
    \textit{No input arguments.}
}{% Example Description
    Show Kane's $\KFrs{r}$ equation for rigid bodies.
}{% Example Text Arguments
    \cs{KaneFrsRig}
}{% Example Function Call
    \KaneFrsRig
}

\entryl{% Command Name
    KaneFr
}{% Arguments
    { }
}{% Description
    Kane's generalized $\KFr{r}$ equation. The command defines Kane's $\KFr{r}$ equation for a general system of $N_{\!\smn{R}}$ rigid bodies and $N_{\!\smn{P}}$ particles.
}{% Argument Descriptions
    \textit{No input arguments.}
}{% Example Description
    Show Kane's generalized $\KFr{r}$ equation.
}{% Example Text Arguments
    \cs{KaneFr}
}{% Example Function Call
    \KaneFr
}

\newpage

\entryl{% Command Name
    KaneFrs
}{% Arguments
    { }
}{% Description
    Kane's generalized $\KFrs{r}$ equation. The command defines Kane's $\KFrs{r}$ equation for a general system of $N_{\!\smn{R}}$ rigid bodies and $N_{\!\smn{P}}$ particles.
}{% Argument Descriptions
    \textit{No input arguments.}
}{% Example Description
    Show Kane's generalized $\KFrs{r}$ equation.
}{% Example Text Arguments
    \cs{KaneFrs}
}{% Example Function Call
    \KaneFrs
}
\index{Kane's equations|)}
\index{equations|)}