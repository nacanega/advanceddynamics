%% The LaTeX package advanceddynamics - version 1.0 (2023/08/24)
%% advdyndoc-04-advanced-terms.tex - Documentation Input File
%%
%% -----------------------------------------------------------------------------
%% Copyright (c) 2023 Nolan Canegallo
%% -----------------------------------------------------------------------------
%%
%% This file may be distributed and/or modified
%%
%%     1.  under the LaTeX Project Public License Version 1.3c and/or
%%     2.  under the GNU Public License Version 3.
%%
%% See the LICENSE.txt for more details.
%%
\section{Advanced Terms}
\label{sec:advterms}
\index{terms!advanced|(}
This section includes the commands needed to typeset most of the terms from \textit{MAE 789}. Note that all commands in this section should be used inside of a math environment.

\subsection*{\blu{Partial Terms}}
\label{sec:partterms}
\index{terms!partial|(}

\entry{% Command Name
    Pvec
}{% Arguments
    \targsiii{frame1}{frame2}{var}
}{% Description
    P-vector. This command is used to express the P-vector of \meta{frame2} relative to \meta{frame1} with respect to \meta{var}.
}{% Argument Descriptions
    \inputsiii{%
        frame1}{Frame; letter (a-z, A-Z)}{%
        frame2}{Frame; letter (a-z, A-Z)}{%
        var}{Variable; symbol}%
}{% Example Description
    Say we want to define the P-vector of the $\fr{B}$ frame relative to the $\fr{A}$ with respect to $\theta$.
}{% Example Text Arguments
    \comarg{Pvec}{\targmiii{a}{b}{\cs{theta}}}
}{% Example Function Call
    \Pvec{a}{b}{\theta}
}

\newpage

\entry{% Command Name
    Pvecx
}{% Arguments
    \targsiii{frame1}{frame2}{var}
}{% Description
    P-vector x-component. This command is used to express the x-component of the P-vector of \meta{frame2} relative to \meta{frame1} with respect to \meta{var}.
}{% Argument Descriptions
    \inputsiii{%
        frame1}{Frame; letter (a-z, A-Z)}{%
        frame2}{Frame; letter (a-z, A-Z)}{%
        var}{Variable; symbol}%
}{% Example Description
    Say we want to define the x-component of the P-vector of the $\fr{B}$ frame relative to the $\fr{A}$ with respect to $\theta$.
}{% Example Text Arguments
    \comarg{Pvecx}{\targmiii{a}{b}{\cs{theta}}}
}{% Example Function Call
    \Pvecx{a}{b}{\theta}
}

\entry{% Command Name
    Pvecy
}{% Arguments
    \targsiii{frame1}{frame2}{var}
}{% Description
    P-vector y-component. This command is used to express the y-component of the P-vector of \meta{frame2} relative to \meta{frame1} with respect to \meta{var}.
}{% Argument Descriptions
    \inputsiii{%
        frame1}{Frame; letter (a-z, A-Z)}{%
        frame2}{Frame; letter (a-z, A-Z)}{%
        var}{Variable; symbol}%
}{% Example Description
    Say we want to define the y-component of the P-vector of the $\fr{B}$ frame relative to the $\fr{A}$ with respect to $\theta$.
}{% Example Text Arguments
    \comarg{Pvecy}{\targmiii{a}{b}{\cs{theta}}}
}{% Example Function Call
    \Pvecy{a}{b}{\theta}
}

\newpage

\entry{% Command Name
    Pvecz
}{% Arguments
    \targsiii{frame1}{frame2}{var}
}{% Description
    P-vector z-component. This command is used to express the z-component of the P-vector of \meta{frame2} relative to \meta{frame1} with respect to \meta{var}.
}{% Argument Descriptions
    \inputsiii{%
        frame1}{Frame; letter (a-z, A-Z)}{%
        frame2}{Frame; letter (a-z, A-Z)}{%
        var}{Variable; symbol}%
}{% Example Description
    Say we want to define the z-component of the P-vector of the $\fr{B}$ frame relative to the $\fr{A}$ with respect to $\theta$.
}{% Example Text Arguments
    \comarg{Pvecz}{\targmiii{a}{b}{\cs{theta}}}
}{% Example Function Call
    \Pvecz{a}{b}{\theta}
}
\index{terms!partial|)}

\subsection*{\blu{Total Terms}}
\label{sec:totalterms}
\index{terms!total|(}

\entry{% Command Name
    Tvec
}{% Arguments
    \targsiii{frame1}{frame2}{var}
}{% Description
    T-vector. This command is used to express the T-vector of \meta{frame2} relative to \meta{frame1} with respect to \meta{var}.
}{% Argument Descriptions
    \inputsiii{%
        frame1}{Frame; letter (a-z, A-Z)}{%
        frame2}{Frame; letter (a-z, A-Z)}{%
        var}{Variable; symbol}%
}{% Example Description
    Say we want to define the T-vector of the $\fr{B}$ frame relative to the $\fr{A}$ with respect to $\theta$.
}{% Example Text Arguments
    \comarg{Tvec}{\targmiii{a}{b}{\cs{theta}}}
}{% Example Function Call
    \Tvec{a}{b}{\theta}
}

\newpage

\entry{% Command Name
    Tvecx
}{% Arguments
    \targsiii{frame1}{frame2}{var}
}{% Description
    T-vector x-component. This command is used to express the x-component of the T-vector of \meta{frame2} relative to \meta{frame1} with respect to \meta{var}.
}{% Argument Descriptions
    \inputsiii{%
        frame1}{Frame; letter (a-z, A-Z)}{%
        frame2}{Frame; letter (a-z, A-Z)}{%
        var}{Variable; symbol}%
}{% Example Description
    Say we want to define the x-component of the T-vector of the $\fr{B}$ frame relative to the $\fr{A}$ with respect to $\theta$.
}{% Example Text Arguments
    \comarg{Tvecx}{\targmiii{a}{b}{\cs{theta}}}
}{% Example Function Call
    \Tvecx{a}{b}{\theta}
}

\entry{% Command Name
    Tvecy
}{% Arguments
    \targsiii{frame1}{frame2}{var}
}{% Description
    T-vector y-component. This command is used to express the y-component of the T-vector of \meta{frame2} relative to \meta{frame1} with respect to \meta{var}.
}{% Argument Descriptions
    \inputsiii{%
        frame1}{Frame; letter (a-z, A-Z)}{%
        frame2}{Frame; letter (a-z, A-Z)}{%
        var}{Variable; symbol}%
}{% Example Description
    Say we want to define the y-component of the T-vector of the $\fr{B}$ frame relative to the $\fr{A}$ with respect to $\theta$.
}{% Example Text Arguments
    \comarg{Tvecy}{\targmiii{a}{b}{\cs{theta}}}
}{% Example Function Call
    \Tvecy{a}{b}{\theta}
}

\newpage

\entry{% Command Name
    Tvecz
}{% Arguments
    \targsiii{frame1}{frame2}{var}
}{% Description
    T-vector z-component. This command is used to express the z-component of the T-vector of \meta{frame2} relative to \meta{frame1} with respect to \meta{var}.
}{% Argument Descriptions
    \inputsiii{%
        frame1}{Frame; letter (a-z, A-Z)}{%
        frame2}{Frame; letter (a-z, A-Z)}{%
        var}{Variable; symbol}%
}{% Example Description
    Say we want to define the z-component of the T-vector of the $\fr{B}$ frame relative to the $\fr{A}$ with respect to $\theta$.
}{% Example Text Arguments
    \comarg{Tvecz}{\targmiii{a}{b}{\cs{theta}}}
}{% Example Function Call
    \Tvecz{a}{b}{\theta}
}
\index{terms!total|)}

\subsection*{\blu{Equations of Motion Terms}}
\label{sec:eomterms}
\index{terms!equations of motion|(}
The terms in this section are used most commonly when deriving equations of motion of systems using more advanced methods such as Lagrange's equations and Kane's equations. 

\entry{% Command Name
    angVk
}{% Arguments
    \targsiii{frame1}{frame2}{k}
}{% Description
    Angular velocity k. This command is used to express the angular velocity of \meta{frame2}-\meta{k} relative to \meta{frame1}.
}{% Argument Descriptions
    \inputsiii{%
        frame1}{First frame; letter (a-z, A-Z)}{%
        frame2}{Second frame; letter (a-z, A-Z)}{%
        k}{Number; positive integer}%
}{% Example Description
    Say we want to define the angular velocity of the $\frn{A}{5}$ frame relative to the $\fr{O}$ frame.
}{% Example Text Arguments
    \comarg{angVk}{\targmiii{O}{A}{5}}
}{% Example Function Call
    \angVk{O}{A}{5}
}

\newpage

\entry{% Command Name
    angAk
}{% Arguments
    \targsiii{frame1}{frame2}{k}
}{% Description
    Angular acceleration k. This command is used to express the angular acceleration of \meta{frame2}-\meta{k} relative to \meta{frame1}.
}{% Argument Descriptions
    \inputsiii{%
        frame1}{First frame; letter (a-z, A-Z)}{%
        frame2}{Second frame; letter (a-z, A-Z)}{%
        k}{Number; positive integer}%
}{% Example Description
    Say we want to define the angular velocity of the $\frn{B}{6}$ frame relative to the $\fr{O}$ frame.
}{% Example Text Arguments
    \comarg{angAk}{\targmiii{o}{b}{6}}
}{% Example Function Call
    \angAk{a}{b}{6}
}

\index{force!internal}
\entry{% Command Name
    forc
}{% Arguments
    \targsi{sub}
}{% Description
    Lowercase force. This command is used to express the forces (usually internal).
}{% Argument Descriptions
    \inputsi{sub}{Subscript; expression}
}{% Example Description
    Say we want to define the force acting on the ith body.
}{% Example Text Arguments
    \comarg{forc}{\targmi{i}}
}{% Example Function Call
    \forc{i}
}

\newpage

\entry{% Command Name
    ptraVr
}{% Arguments
    \targsiii{frame}{point}{r}
}{% Description
    r-th partial velocity. This command is used to express rth partial velocity of \meta{point} in \meta{frame}.
}{% Argument Descriptions
    \inputsiii{%
        frame}{Frame; letter (a-z, A-Z)}{%
        point}{Point; expression}{%
        r}{Number; positive integer}%
}{% Example Description
    Say we want to define the 4th partial velocity of $m_i$ in $\fr{O}$.
}{% Example Text Arguments
    \comarg{ptraVr}{\targmiii{o}{m\string_i}{4}}
}{% Example Function Call
    \ptraVr{a}{m_i}{4}
}

\entry{% Command Name
    ptraVt
}{% Arguments
    \targsii{frame}{point}
}{% Description
    time partial velocity. This command is used to express the time partial velocity of \meta{point} in \meta{frame}.
}{% Argument Descriptions
    \inputsii{%
        frame}{Frame; letter (a-z, A-Z)}{%
        point}{Point; expression}%
}{% Example Description
    Say we want to define the time partial velocity of $m_i$ in $\fr{O}$.
}{% Example Text Arguments
    \comarg{ptraVt}{\targmii{o}{m\string_i}}
}{% Example Function Call
    \ptraVt{a}{m_i}
}

\newpage

\entry{% Command Name
    pangVr
}{% Arguments
    \targsiii{frame1}{frame2}{r}
}{% Description
    r-th partial angular velocity. This command is used to express rth partial angular velocity of \meta{frame2} in \meta{frame1}.
}{% Argument Descriptions
    \inputsiii{%
        frame1}{Frame; letter (a-z, A-Z)}{%
        frame2}{Frame; letter (a-z, A-Z)}{%
        r}{Number; positive integer}%
}{% Example Description
    Say we want to define the 4th partial angular velocity of $\fr{B}$ in $\fr{A}$.
}{% Example Text Arguments
    \comarg{pangVr}{\targmiii{a}{b}{4}}
}{% Example Function Call
    \pangVr{a}{b}{4}
}

\entry{% Command Name
    pangVrk
}{% Arguments
    \targsiv{frame1}{frame2}{k}{r}
}{% Description
    r-th partial angular velocity k. This command is used to express rth partial angular velocity of \meta{frame2}-\meta{k} in \meta{frame1}.
}{% Argument Descriptions
    \inputsiv{%
        frame1}{Frame; letter (a-z, A-Z)}{%
        frame2}{Frame; letter (a-z, A-Z)}{%
        k}{Number; positive integer}{%
        r}{Number; positive integer}%
}{% Example Description
    Say we want to define the 4th partial angular velocity of $\frn{B}{2}$ in $\fr{A}$.
}{% Example Text Arguments
    \comarg{pangVr}{\targmiv{a}{b}{2}{4}}
}{% Example Function Call
    \pangVrk{a}{b}{2}{4}
}

\newpage

\entry{% Command Name
    pangVt
}{% Arguments
    \targsii{frame1}{frame2}
}{% Description
    time partial angular velocity. This command is used to express the time partial angular velocity of \meta{frame2} in \meta{frame1}.
}{% Argument Descriptions
    \inputsii{%
        frame1}{Frame; letter (a-z, A-Z)}{%
        frame2}{Frame; letter (a-z, A-Z)}%
}{% Example Description
    Say we want to define the time partial angular velocity of $\fr{b}$ in $\fr{a}$.
}{% Example Text Arguments
    \comarg{pangVt}{\targmii{o}{b}}
}{% Example Function Call
    \pangVt{a}{b}
}

\index{Lagrange's equations!Lagrangian}
\entry{% Command Name
    Lagr
}{% Arguments
    { }
}{% Description
    Lagrangian. Show the symbol for the Lagrangian.
}{% Argument Descriptions
    \textit{No input arguments.}
}{% Example Description
    Show the symbol for the Lagrangian.
}{% Example Text Arguments
    \cs{Lagr}
}{% Example Function Call
    \Lagr
}

\newpage

\index{Kane's equations!Fr}
\entry{% Command Name
    KFr
}{% Arguments
    \targsi{r}
}{% Description
    $\KFr{r}$. This command is used to express $\KFr{r}$ from Kane's equations of motion.
}{% Argument Descriptions
    \inputsi{r}{Number; positive integer}
}{% Example Description
    Say we want to define $\KFr{1}$.
}{% Example Text Arguments
    \comarg{KFr}{\targmi{1}}
}{% Example Function Call
    \KFr{1}
}

\index{Kane's equations!Fr-star}
\entry{% Command Name
    KFrs
}{% Arguments
    \targsi{r}
}{% Description
    $\KFrs{r}$. This command is used to express $\KFrs{r}$ from Kane's equations of motion.
}{% Argument Descriptions
    \inputsi{r}{r; positive integer}
}{% Example Description
    Say we want to define $\KFrs{1}$.
}{% Example Text Arguments
    \comarg{KFrs}{\targmi{1}}
}{% Example Function Call
    \KFrs{1}
}
\index{terms!equations of motion|)}
\index{terms!advanced|)}