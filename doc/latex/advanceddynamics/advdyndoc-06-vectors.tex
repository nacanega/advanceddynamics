%% The LaTeX package advanceddynamics - version 1.0 (2023/08/24)
%% advdyndoc-06-vectors.tex - Documentation Input File
%%
%% -----------------------------------------------------------------------------
%% Copyright (c) 2023 Nolan Canegallo
%% -----------------------------------------------------------------------------
%%
%% This file may be distributed and/or modified
%%
%%     1.  under the LaTeX Project Public License Version 1.3c and/or
%%     2.  under the GNU Public License Version 3.
%%
%% See the LICENSE.txt for more details.
%%
\section{Vectors}
\label{sec:vecs}
\index{vectors|(}
This section includes the commands needed to typeset most vector quantities. Note that all commands in this section should be used inside of a math environment. Vectors should be in a display math environment. 

Note that if you wish to attach a frame subscipt to any of the elementary vectors, you should also include \cs{!} or \cs{!}\cs{!} before the frame definition to remove the space between the closing bracket of the vector and the frame. This will be addressed in a future version.

\subsection*{\blu{Column Vectors}}
\label{sec:colvecs}
\index{vectors!column|(}
\entryl{% Command Name
    vecfrc
}{% Arguments
    \targsiv{frame}{comp1}{comp2}{comp3}
}{% Description
    Column vector in frame. This command produces a column vector in \meta{frame} with x-component \meta{comp1}, y-component \meta{comp2}, and z-component \meta{comp3}.
}{% Argument Descriptions
    \inputsiv{%
        frame}{Frame; letter (a-z, A-Z)}{%
        comp1}{x-component; expression}{%
        comp2}{y-component; expression}{%
        comp3}{z-component; expression}
}{% Example Description
    Say we want to define a column vector in the $\fr{F}$ frame with components $a$, $b$, and $c$ in the $\ihat{f}$, $\jhat{f}$, and $\khat{f}$ directions, respectively.
}{% Example Text Arguments
    \comarg{vecfrc}{\targmiv{f}{a}{b}{c}}
}{% Example Function Call
    \vecfrc{f}{a}{b}{c}
}

\newpage

\entryl{% Command Name
    vecfrnc
}{% Arguments
    \targsv{frame}{num}{comp1}{comp2}{comp3}
}{% Description
    Column vector in numbered frame. This command produces a column vector in \meta{frame} assigned number \meta{num}, with x-component \meta{comp1}, y-component \meta{comp2}, and z-component \meta{comp3}.
}{% Argument Descriptions
    \inputsv{%
        frame}{Frame; letter (a-z, A-Z)}{%
        num}{Frame; letter (a-z, A-Z)}{%
        comp1}{x-component; expression}{%
        comp2}{y-component; expression}{%
        comp3}{z-component; expression}
}{% Example Description
    Say we want to define a column vector in the $\frn{F}{2}$ frame with components $a$, $b$, and $c$ in the $\ihat{f}$, $\jhat{f}$, and $\khat{f}$ directions, respectively.
}{% Example Text Arguments
    \comarg{vecfrnc}{\targmv{f}{2}{a}{b}{c}}
}{% Example Function Call
    \vecfrnc{f}{2}{a}{b}{c}
}

\entryl{% Command Name
    vecAc
}{% Arguments
    { }
}{% Description
    First elementary column vector. This command is used to express the elementary column vector along the x-direction of a frame.
}{% Argument Descriptions
    \textit{No input arguments.}
}{% Example Description
    Define the elementary column unit vector for the x-component.
}{% Example Text Arguments
    \comarg{vecAc}
}{% Example Function Call
    \vecAc
}

\newpage

\entryl{% Command Name
    vecBc
}{% Arguments
    { }
}{% Description
    Second elementary column vector. This command is used to express the elementary column vector along the y-direction of a frame.
}{% Argument Descriptions
    \textit{No input arguments.}
}{% Example Description
    Define the elementary column unit vector for the y-component.
}{% Example Text Arguments
    \comarg{vecBc}
}{% Example Function Call
    \vecBc
}

\entryl{% Command Name
    vecCc
}{% Arguments
    { }
}{% Description
    Third elementary column vector. This command is used to express the elementary column vector along the z-direction of a frame.
}{% Argument Descriptions
    \textit{No input arguments.}
}{% Example Description
    Define the elementary column unit vector for the z-component.
}{% Example Text Arguments
    \comarg{vecCc}
}{% Example Function Call
    \vecCc
}
\index{vectors!column|)}

\newpage

\subsection*{\blu{Transposed Column Vectors}}
\label{sec:colvecsT}
\index{vectors!transposed column|(}

\entryd{% Command Name
    vecfrcT
}{% Arguments
    \targsiv{frame}{comp1}{comp2}{comp3}
}{% Description
    Column vector in frame. This command produces a transposed column vector in \meta{frame} with x-component \meta{comp1}, y-component \meta{comp2}, and z-component \meta{comp3}.
}{% Argument Descriptions
    \inputsiv{%
        frame}{Frame; letter (a-z, A-Z)}{%
        comp1}{x-component; expression}{%
        comp2}{y-component; expression}{%
        comp3}{z-component; expression}
}{% Example Description
    Say we want to define a transposed column vector in the $\fr{F}$ frame with components $a$, $b$, and $c$ in the $\ihat{f}$, $\jhat{f}$, and $\khat{f}$ directions, respectively.
}{% Example Text Arguments
    \comarg{vecfrcT}{\targmiv{f}{a}{b}{c}}
}{% Example Function Call
    \vecfrcT{f}{a}{b}{c}
}

\newpage

\entryd{% Command Name
    vecfrncT
}{% Arguments
    \targsv{frame}{num}{comp1}{comp2}{comp3}
}{% Description
    Column vector in numbered frame. This command produces a transposed column vector in \meta{frame} assigned number \meta{num}, with x-component \meta{comp1}, y-component \meta{comp2}, and z-component \meta{comp3}.
}{% Argument Descriptions
    \inputsv{%
        frame}{Frame; letter (a-z, A-Z)}{%
        num}{Frame; letter (a-z, A-Z)}{%
        comp1}{x-component; expression}{%
        comp2}{y-component; expression}{%
        comp3}{z-component; expression}
}{% Example Description
    Say we want to define a transposed column vector in the $\frn{F}{2}$ frame with components $a$, $b$, and $c$ in the $\ihat{f}$, $\jhat{f}$, and $\khat{f}$ directions, respectively.
}{% Example Text Arguments
    \comarg{vecfrncT}{\targmv{f}{2}{a}{b}{c}}
}{% Example Function Call
    \vecfrncT{f}{2}{a}{b}{c}
}

\entryd{% Command Name
    vecAcT
}{% Arguments
    { }
}{% Description
    First transposed elementary column vector. This command is used to express the transposed elementary column vector along the x-direction of a frame.
}{% Argument Descriptions
    \textit{No input arguments.}
}{% Example Description
    Define the transposed elementary unit vector for the x-component.
}{% Example Text Arguments
    \comarg{vecAcT}
}{% Example Function Call
    \vecAcT
}

\newpage

\entryd{% Command Name
    vecBcT
}{% Arguments
    { }
}{% Description
    Second transposed elementary column vector. This command is used to express the transposed elementary column vector along the y-direction of a frame.
}{% Argument Descriptions
    \textit{No input arguments.}
}{% Example Description
    Define the transposed elementary unit vector for the y-component.
}{% Example Text Arguments
    \comarg{vecBcT}
}{% Example Function Call
    \vecBcT
}

\entryd{% Command Name
    vecCcT
}{% Arguments
    { }
}{% Description
    Third transposed elementary column vector. This command is used to express the transposed elementary column vector along the z-direction of a frame.
}{% Argument Descriptions
    \textit{No input arguments.}
}{% Example Description
    Define the transposed elementary unit vector for the z-component.
}{% Example Text Arguments
    \comarg{vecCcT}
}{% Example Function Call
    \vecCcT
}
\index{vectors!transposed column|)}

\newpage

\subsection*{\blu{Row Vectors}}
\label{sec:rowvecs}
\index{vectors!row|(}

\entryd{% Command Name
    vecfrr
}{% Arguments
    \targsiv{frame}{comp1}{comp2}{comp3}
}{% Description
    Row vector in frame. This command produces a row vector in \meta{frame} with x-component \meta{comp1}, y-component \meta{comp2}, and z-component \meta{comp3}.
}{% Argument Descriptions
    \inputsiv{%
        frame}{Frame; letter (a-z, A-Z)}{%
        comp1}{x-component; expression}{%
        comp2}{y-component; expression}{%
        comp3}{z-component; expression}
}{% Example Description
    Say we want to define a row vector in the $\fr{F}$ frame with components $a$, $b$, and $c$ in the $\ihat{f}$, $\jhat{f}$, and $\khat{f}$ directions, respectively.
}{% Example Text Arguments
    \comarg{vecfrr}{\targmiv{f}{a}{b}{c}}
}{% Example Function Call
    \vecfrr{f}{a}{b}{c}
}

\newpage

\entryd{% Command Name
    vecfrnr
}{% Arguments
    \targsv{frame}{num}{comp1}{comp2}{comp3}
}{% Description
    Row vector in numbered frame. This command produces a row vector in \meta{frame} assigned number \meta{num}, with x-component \meta{comp1}, y-component \meta{comp2}, and z-component \meta{comp3}.
}{% Argument Descriptions
    \inputsv{%
        frame}{Frame; letter (a-z, A-Z)}{%
        num}{Frame; letter (a-z, A-Z)}{%
        comp1}{x-component; expression}{%
        comp2}{y-component; expression}{%
        comp3}{z-component; expression}
}{% Example Description
    Say we want to define a row vector in the $\frn{F}{2}$ frame with components $a$, $b$, and $c$ in the $\ihat{f}$, $\jhat{f}$, and $\khat{f}$ directions, respectively.
}{% Example Text Arguments
    \comarg{vecfrnr}{\targmv{f}{2}{a}{b}{c}}
}{% Example Function Call
    \vecfrnr{f}{2}{a}{b}{c}
}

\entryd{% Command Name
    vecAr
}{% Arguments
    { }
}{% Description
    First elementary row vector. This command is used to express the elementary row vector along the x-direction of a frame.
}{% Argument Descriptions
    \textit{No input arguments.}
}{% Example Description
    Define the elementary row unit vector for the x-component.
}{% Example Text Arguments
    \comarg{vecAr}
}{% Example Function Call
    \vecAr
}

\newpage

\entryd{% Command Name
    vecBr
}{% Arguments
    { }
}{% Description
    Second elementary row vector. This command is used to express the elementary row vector along the y-direction of a frame.
}{% Argument Descriptions
    \textit{No input arguments.}
}{% Example Description
    Define the elementary row unit vector for the y-component.
}{% Example Text Arguments
    \comarg{vecBr}
}{% Example Function Call
    \vecBr
}

\entryd{% Command Name
    vecCr
}{% Arguments
    { }
}{% Description
    Third elementary row vector. This command is used to express the elementary row vector along the z-direction of a frame.
}{% Argument Descriptions
    \textit{No input arguments.}
}{% Example Description
    Define the elementary row unit vector for the z-component.
}{% Example Text Arguments
    \comarg{vecCr}
}{% Example Function Call
    \vecCr
}
\index{vectors!row|)}
\index{vectors|)}