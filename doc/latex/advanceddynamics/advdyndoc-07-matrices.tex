%% The LaTeX package advanceddynamics - version 1.0 (2023/08/24)
%% advdyndoc-07-matrices.tex - Documentation Input File
%%
%% -----------------------------------------------------------------------------
%% Copyright (c) 2023 Nolan Canegallo
%% -----------------------------------------------------------------------------
%%
%% This file may be distributed and/or modified
%%
%%     1.  under the LaTeX Project Public License Version 1.3c and/or
%%     2.  under the GNU Public License Version 3.
%%
%% See the LICENSE.txt for more details.
%%
\section{Matrices}
\label{sec:matrices}
\index{matrix|see {rotation matrix}}
\index{matrix!definitions|(}
This section includes the commands needed to typeset most matrix quantities. Note that all commands in this section should be used inside of a math environent and a display math environment should be used for the best appearance.

Note that if you wish to attach a frame subscipt to any of the matrices without one, you should also include \cs{!} or \cs{!}\cs{!} before the frame definition to remove the space between the closing bracket of the vector and the frame. This will be addressed in a future version.

\subsection*{\blu{Standard Rotation Matrices}}
\label{sec:rotmats}
\index{rotation matrix|(}

\index{rotation matrix!x}
\index{rotation matrix!x}
\entryl{% Command Name
    rotx
}{% Arguments
    \targsi{angle}
}{% Description
    Rotation matrix about x. This command writes all the terms of a rotation matrix about the x-axis using a specified angle, \meta{angle}.
}{% Argument Descriptions
    \inputsi{angle}{Rotation angle; symbol}
}{% Example Description
    Say we want to define the rotation matrix about the x-axis using the angle $\phi$.
}{% Example Text Arguments
    \comarg{rotx}{\targmi{\cs{phi}}}
}{% Example Function Call
    \rotx{\phi}
}

\newpage

\index{rotation matrix!y}
\index{rotation matrix!y}
\entryl{% Command Name
    roty
}{% Arguments
    \targsi{angle}
}{% Description
    Rotation matrix about y. This command writes all the terms of a rotation matrix about the y-axis using a specified angle, \meta{angle}.
}{% Argument Descriptions
    \inputsi{angle}{Rotation angle; symbol}
}{% Example Description
    Say we want to define the rotation matrix about the y-axis using the angle $\theta$.
}{% Example Text Arguments
    \comarg{roty}{\targmi{\cs{theta}}}
}{% Example Function Call
    \roty{\theta}
}

\index{rotation matrix!z}
\index{rotation matrix!z}
\entryl{% Command Name
    rotz
}{% Arguments
    \targsi{angle}
}{% Description
    Rotation matrix about z. This command writes all the terms of a rotation matrix about the z-axis using a specified angle, \meta{angle}.
}{% Argument Descriptions
    \inputsi{angle}{Rotation angle; symbol}
}{% Example Description
    Say we want to define the rotation matrix about the z-axis using the angle $\psi$.
}{% Example Text Arguments
    \comarg{rotz}{\targmi{\cs{psi}}}
}{% Example Function Call
    \rotz{\psi}
}

\newpage

\index{rotation matrix!x, q-shorthand}
\index{rotation matrix!x, q-shorthand}
\entryl{% Command Name
    rotxq
}{% Arguments
    \targsi{num}
}{% Description
    Rotation matrix about x using q shorthand. This command writes all the terms of a rotation matrix about the x-axis using a specified $q_n$, $n$ = \meta{num} and shorthand notation.
}{% Argument Descriptions
    \inputsi{num}{Variable number; positive integer}
}{% Example Description
    Say we want to define the rotation matrix about the x-axis using the variable $q_1$.
}{% Example Text Arguments
    \comarg{rotxq}{\targmi{\code{1}}}
}{% Example Function Call
    \rotxq{1}
}

\index{rotation matrix!y, q-shorthand}
\index{rotation matrix!y, q-shorthand}
\entryl{% Command Name
    rotyq
}{% Arguments
    \targsi{num}
}{% Description
    Rotation matrix about y using q shorthand. This command writes all the terms of a rotation matrix about the y-axis using a specified $q_n$, $n$ = \meta{num} and shorthand notation.
}{% Argument Descriptions
    \inputsi{num}{Variable number; positive integer}
}{% Example Description
    Say we want to define the rotation matrix about the y-axis using the variable $q_2$.
}{% Example Text Arguments
    \comarg{rotyq}{\targmi{\code{2}}}
}{% Example Function Call
    \rotyq{2}
}

\newpage

\index{rotation matrix!z, q-shorthand}
\index{rotation matrix!z, q-shorthand}
\entryl{% Command Name
    rotzq
}{% Arguments
    \targsi{num}
}{% Description
    Rotation matrix about z using q shorthand. This command writes all the terms of a rotation matrix about the z-axis using a specified $q_n$, $n$ = \meta{num} and shorthand notation.
}{% Argument Descriptions
    \inputsi{num}{Variable number; positive integer}
}{% Example Description
    Say we want to define the rotation matrix about the z-axis using the variable $q_3$.
}{% Example Text Arguments
    \comarg{rotzq}{\targmi{\code{3}}}
}{% Example Function Call
    \rotzq{3}
}
\index{rotation matrix|)}

\subsection*{\blu{Transposed/Inverse Rotation Matrices}}
\label{sec:rotmatsT}
\index{rotation matrix!inverse|(}
\index{rotation matrix!transposed|(}

\index{rotation matrix!inverse!x}
\index{rotation matrix!transpose!x}
\entryl{% Command Name
    rotxT
}{% Arguments
    \targsi{angle}
}{% Description
    Transposed rotation matrix about x. This command writes all the terms of a transposed/inverse rotation matrix about the x-axis using a specified angle, \meta{angle}.
}{% Argument Descriptions
    \inputsi{angle}{Rotation angle; symbol}
}{% Example Description
    Say we want to define the transposed/inverse rotation matrix about the x-axis using the angle $\phi$.
}{% Example Text Arguments
    \comarg{rotxT}{\targmi{\cs{phi}}}
}{% Example Function Call
    \rotxT{\phi}
}

\newpage

\index{rotation matrix!inverse!y}
\index{rotation matrix!transpose!y}
\entryl{% Command Name
    rotyT
}{% Arguments
    \targsi{angle}
}{% Description
    Transposed rotation matrix about y. This command writes all the terms of a transposed/inverse rotation matrix about the y-axis using a specified angle, \meta{angle}.
}{% Argument Descriptions
    \inputsi{angle}{Rotation angle; symbol}
}{% Example Description
    Say we want to define the transposed/inverse rotation matrix about the y-axis using the angle $\theta$.
}{% Example Text Arguments
    \comarg{rotyT}{\targmi{\cs{theta}}}
}{% Example Function Call
    \rotyT{\theta}
}

\index{rotation matrix!inverse!z}
\index{rotation matrix!transpose!z}
\entryl{% Command Name
    rotzT
}{% Arguments
    \targsi{angle}
}{% Description
    Transposed rotation matrix about z. This command writes all the terms of a transposed/inverse rotation matrix about the z-axis using a specified angle, \meta{angle}.
}{% Argument Descriptions
    \inputsi{angle}{Rotation angle; symbol}
}{% Example Description
    Say we want to define the transposed/inverse rotation matrix about the z-axis using the angle $\psi$.
}{% Example Text Arguments
    \comarg{rotzT}{\targmi{\cs{psi}}}
}{% Example Function Call
    \rotzT{\psi}
}

\newpage

\index{rotation matrix!inverse!x, q-shorthand}
\index{rotation matrix!transpose!x, q-shorthand}
\entryl{% Command Name
    rotxqT
}{% Arguments
    \targsi{num}
}{% Description
    Transposed rotation matrix about x using q shorthand. This command writes all the terms of a transposed/inverse rotation matrix about the x-axis using a specified $q_n$, $n$ = \meta{num} and shorthand notation.
}{% Argument Descriptions
    \inputsi{num}{Variable number; positive integer}
}{% Example Description
    Say we want to define the transposed/inverse rotation matrix about the x-axis using the variable $q_1$.
}{% Example Text Arguments
    \comarg{rotxqT}{\targmi{\code{1}}}
}{% Example Function Call
    \rotxqT{1}
}

\index{rotation matrix!inverse!y, q-shorthand}
\index{rotation matrix!transpose!y, q-shorthand}
\entryl{% Command Name
    rotyqT
}{% Arguments
    \targsi{num}
}{% Description
    Transposed rotation matrix about y using q shorthand. This command writes all the terms of a transposed/inverse rotation matrix about the y-axis using a specified $q_n$, $n$ = \meta{num} and shorthand notation.
}{% Argument Descriptions
    \inputsi{num}{Variable number; positive integer}
}{% Example Description
    Say we want to define the transposed/inverse rotation matrix about the y-axis using the variable $q_2$.
}{% Example Text Arguments
    \comarg{rotyqT}{\targmi{\code{2}}}
}{% Example Function Call
    \rotyqT{2}
}

\newpage

\index{rotation matrix!inverse!z, q-shorthand}
\index{rotation matrix!transpose!z, q-shorthand}
\entryl{% Command Name
    rotzqT
}{% Arguments
    \targsi{num}
}{% Description
    Transposed rotation matrix about z using q shorthand. This command writes all the terms of a transposed/inverse rotation matrix about the z-axis using a specified $q_n$, $n$ = \meta{num} and shorthand notation.
}{% Argument Descriptions
    \inputsi{num}{Variable number; positive integer}
}{% Example Description
    Say we want to define the transposed/inverse rotation matrix about the z-axis using the variable $q_3$.
}{% Example Text Arguments
    \comarg{rotzqT}{\targmi{\code{3}}}
}{% Example Function Call
    \rotzqT{3}
}
\index{rotation matrix!inverse|)}
\index{rotation matrix!transposed|)}

\subsection*{\blu{Inertia Matrices}}
\label{sec:inermats}

\index{inertia!matrix}
\index{matrix!inertia}
\entryl{% Command Name
    inertiaMat
}{% Arguments
    \targsii{frame}{point}
}{% Description
    Inertia matrix. This command is used to define the terms of an inertia tensor of a body about point \meta{point} in matrix form with body frame \meta{frame}.
}{% Argument Descriptions
    \inputsii{%
        frame}{Frame; letter (a-z, A-Z)}{%
        point}{Point; expression}%
}{% Example Description
    Say we want to define the inertia matrix of body with body frame $\fr{B}$ about its center of mass.
}{% Example Text Arguments
    \comarg{inertiaMat}{\targmii{b}{\cs{CM}}}
}{% Example Function Call
    \inertiaMat{b}{\CM}
}

\newpage

\index{matrix!inertia!shorthand}
\index{inertia!matrix!shorthand}
\entryl{% Command Name
    inertiaMatsh
}{% Arguments
    { }
}{% Description
    Shorthand inertia matrix. This command is used to define the terms of an inertia tensor in shorthand.
}{% Argument Descriptions
    \textit{No input arguments.}
}{% Example Description
    Define the shorthand inertia matrix.
}{% Example Text Arguments
    \cs{inertiaMatsh}
}{% Example Function Call
    \inertiaMatsh
}

\index{matrix!inertia difference}
\index{inertia!difference matrix}
\entryl{% Command Name
    inertiaDif
}{% Arguments
    \targsii{frame}{point}
}{% Description
    Inertia difference matrix. This command is used to define the terms of an inertia difference tensor of a body about point \meta{point} relative to its center of mass in matrix form with body frame \meta{frame}.
}{% Argument Descriptions
    \inputsii{%
        frame}{Frame; letter (a-z, A-Z)}{%
        point}{Point; expression}%
}{% Example Description
    Say we want to define the inertia difference matrix of body with body frame $\fr{B}$ about a point $A$.
}{% Example Text Arguments
    \comarg{inertiaDif}{\targmii{b}{\comone{smca}{a}}}
}{% Example Function Call
    \inertiaDif{b}{\smca{a}}
}

\newpage

\index{inertia!difference matrix!shorthand}
\index{matrix!inertia difference!shorthand}
\entryl{% Command Name
    inertiaDifsh
}{% Arguments
    { }
}{% Description
    Shorthand inertia difference matrix. This command is used to define the terms of an inertia difference tensor relative to its center of mass in shorthand.
}{% Argument Descriptions
    \textit{No input arguments.}
}{% Example Description
    Define the shorthand inertia difference matrix.
}{% Example Text Arguments
    \cs{inertiaDifsh}
}{% Example Function Call
    \inertiaDifsh
}

\subsection*{\blu{Other Matrices}}
\label{sec:othermats}
\index{matrix!other|(}

\index{matrix!tensor}
\index{tensor!matrix}
\entryl{% Command Name
    tensorMat
}{% Arguments
    \targsii{frame}{tensor}
}{% Description
    Tensor matrix. This command is used to define the terms of a matrix representation of the tensor \meta{tensor} relative to the frame \meta{frame}.
}{% Argument Descriptions
    \inputsii{%
        frame}{Frame; letter (a-z, A-Z)}{%
        tensor}{Tensor; expression}
}{% Example Description
    Say we want to define a matrix representation of an inertia tensor about a body's center of mass expressed in its body frame $\fr{B}$.
}{% Example Text Arguments
    \comarg{tensorMat}{\targmii{b}{\comone{inerTen}{\cs{CM}}}}
}{% Example Function Call
    \tensorMat{b}{\inerTen{\CM}}
}

\newpage

\index{matrix!square}
\index{square!matrix}
\entryl{% Command Name
    sqMatiii
}{% Arguments
    \targsix{A11}{A12}{A13}{A21}{A22}{A23}{A31}{A32}{A33}
}{% Description
    3x3 Matrix. This command is used to define a 3x3 square matrix with specified terms.
}{% Argument Descriptions
    \inputsi{%
        Amn}{(m,n) term; expression}%
}{% Example Description
    Say we want to define a 3x3 matrix with the values 1 to 9 ascending from left to right, top to bottom.
}{% Example Text Arguments
    \comarg{sqMatiii}{\targmix{1}{2}{3}{4}{5}{6}{7}{8}{9}}
}{% Example Function Call
    \sqMatiii{1}{2}{3}{4}{5}{6}{7}{8}{9}
}

\index{matrix!identity}
\index{identity!matrix}
\entryl{% Command Name
    eyeMatiii
}{% Arguments
    { }
}{% Description
    3x3 identity matrix.
}{% Argument Descriptions
    \textit{No input arguments.}
}{% Example Description
    Define the 3x3 identity matrix.
}{% Example Text Arguments
    \cs{eyeMatiii}
}{% Example Function Call
    \eyeMatiii
}

\newpage

\index{matrix!direction cosine}
\entryl{% Command Name
    dcm
}{% Arguments
    { }
}{% Description
    Direction cosine matrix. This command defines the terms of a direction cosine matrix.
}{% Argument Descriptions
    \textit{No input arguments.}
}{% Example Description
    Define the direction cosine matrix.
}{% Example Text Arguments
    \cs{dcm}
}{% Example Function Call
    \dcm
}

\index{matrix!cross!definition}
\entryl{% Command Name
    crossMat
}{% Arguments
    \targsiii{xcomp}{ycomp}{zcomp}
}{% Description
    Cross matrix. This command is used to define a cross matrix which when multiplied by a vector is equivalent to taking a cross product. Defined given the x, y, and z components of the first vector in the cross product.
}{% Argument Descriptions
    \inputsiii{%
        xcomp}{x-component; expression}{%
        ycomp}{y-component; expression}{%
        zcomp}{z-component; expression}%
}{% Example Description
    Say we want to define a cross matrix for a vector $\vec{r} = [x\ y\ z]$.
}{% Example Text Arguments
    \comarg{crossMat}{\targmiii{x}{y}{z}}
}{% Example Function Call
    \crossMat{x}{y}{z}
}
\index{matrix!other|)}
\index{matrix!definitions|)}